% TiMBL 4.2 API

\documentclass{article}
\usepackage{epsf}
\usepackage{a4wide}
\usepackage{palatino}
\usepackage{fullname}

\parindent 0pt
\parskip 9pt

\newcommand{\chisq}{{$ \chi^2 $}}

\author{Ko van der Sloot\\ \ \\ Induction of Linguistic Knowledge\\
        Computational Linguistics\\ Tilburg University \\ \ \\
        P.O. Box 90153, NL-5000 LE, Tilburg, The Netherlands \\ URL:
        http://ilk.kub.nl}

\title{{\huge TiMBL: Tilburg Memory-Based Learner} \\ \vspace*{0.5cm}
{\bf version 4.2} \\ \vspace*{0.5cm}{\huge API Reference Guide}\\
\vspace*{1cm} {\it ILK Technical Report -- ILK 02-01a}}


\begin{document}

\maketitle

\clearpage
\section{Preface}

This is brief description of the TimblAPI class and its main functions.
Not everything found in TimblAPI.h is described. Some functions are
still "work in progress" and some others are artefacts to simplify the
implementation of the Timbl main program\footnote{Timbl.cxx is
therefore {\em not} a good example of how to use the API.}. \\
To learn more about using the API, you should study programs like 
{\tt classify.cxx, tse.cxx} and of course the examples given in this
manual, which can be found in the {\tt demos} directory of this
distribution. 

As you can learn from these examples, all you need to get access to the
TimblAPI functions, is to include TimblAPI.h in the program, and to
include libTimbl.a in your linking path. 

Important note: All described functions return a bool to indicate
succes or failure. To simplify the examples, we ignore these return
values. This is, of course, bad practice, to be avoided in
real life programming.

\section{Starting our first experiment}

There is just one way to start a Timbl experiment, and that is by
calling the TimblAPI constructor:

\begin{verbatim}
TimblAPI( const char *args, const char *name=NULL );
\end{verbatim}

args is used as a "command line" and is parsed for all kind of
options which are used to create the right kind of experiment with the
desired settings for metric, weighting etc.\\
If something is wrong with the settings, {\em no} object is created.

The most important option is ``-a''  to set the kind of algorithm,
e.g. "-a IB1" to get an IB1 experiment or "-a IGTREE" to get an IGTREE
experiment. A list of possible options is give in Appendix A.\\

The optional name can be useful if you have multiple experiments.
In case of warnings or errors, this name is appended to the message.\\

\paragraph{Example:}\ \\
\begin{verbatim}
  TimblAPI *My_Experiment = new TimblAPI( "-a IGTREE +vDI+DB", 
                                          "test1" );
\end{verbatim}

My\_Experiment is created as an IGTREE experiment with the name "test1"
and the verbosity is set to DI+DB, meaning that the output will
contain DIstance and DistriBution information.

Of course there is a counterpart to creation in the form of the 
\~{ }TimblAPI() destructor, which is called when you delete an
experiment with:

\begin{verbatim}
  delete My_Experiment;
\end{verbatim}

\section{More about settings}

After an experiment is set up with the TimblAPI constructor, many
options can be changed "on the fly" with:

\begin{verbatim}
  bool SetOptions( const char *opts );
\end{verbatim}

'opts' is interpreted as a list of options which are set, just like in
the TimblAPI constructor. When an error in the opts string is found,
SetOptions() returns false. Whether any options are really set or
changed in that case is undefined. \\
Note that a few options can only be set {\em once} when creating the
experiment. Most notably the algorithm. Any attempt to change these
options will result in a failure.
See Appendix A for all valid options and information about the
possibility to change them for a running experiment.

Note: SetOptions() is performed "lazy", changes are cached until the
moment they are really needed, so you can do several SetOptions()
calls with even different values for the same option. Only the last
one seen will be used for running the experiment.

To see which options are in effect, you can use the calls ShowOptions()
and ShowSettings().

\begin{verbatim}
  bool ShowOptions( std::ostream& );
\end{verbatim}

Shows all options with their possible and current values.

\begin{verbatim}
  bool ShowSettings( std::ostream& );
\end{verbatim}

Shows all options and their currect values.\\

\paragraph{Example:}\ \\

\begin{verbatim}
  My_Experiment->SetOptions( "-w2 -m:M" );
  My_Experiment->SetOptions( "-w3 -v:DB" );
  My_Experiment->ShowSettings( cout )
\end{verbatim}

See Appendix B for the output.

\section{Running our first experiment}

Assuming that we have appropriate datafiles, (such as {\tt
dimin.train} and {\tt dimin.test} in the Timbl package) we can get
started quite easy with the functions Learn() and Test().

\subsection{Training}
\begin{verbatim}
  bool Learn( const char *f );
\end{verbatim}

This function takes a file with name f, and gathers information like: number
of features, number and frequency of feature values and the same for
class names. \\
After that, these data are used to calculate a lot of statistical
information, which will be used for testing. Finally, an InstanceBase
is created, tuned to the current algorithm.\\ 

Note: There are special cases where Learn() behaves differently:

\begin{itemize}
\item When the algorithm is IB2, Learn() will automatically take the
first 'n' lines of f (set with the -b n option) to bootstrap itself,
and then the rest of f for IB2-learning. 

After Learning IB2, you can use Test().

\item When the algorithm is CV, Learn() is not defined, and all work
is done in a special version of Test().
\end{itemize}

\subsection{Testing}
\begin{verbatim}
  bool Test( const char *in,
             const char *out,
             const char *perc = NULL );
\end{verbatim}

Test a file given by 'in' and write results to 'out'. If 'perc' isn't
NULL, then a percentage score is written to file 'perc'.

\paragraph{Example:}
\begin{verbatim}
  My_Experiment->Learn( "dimin.train" );  
  My_Experiment->Test( "dimin.test", "my_first_test" );  
\end{verbatim}

An InstanceBase will be created from dimin.train, then dimin.test is
tested against that InstanceBase and output is written to
my\_first\_test.

\ \\
For Crossvalidation, there is a special implementation for Test(f).\\
'f' is assumed to give the name of a file, which, on separate lines,
gives the names of the files to be cross-validated. 

See Appendix B for a complete example.

\section{Storing and retrieving intermediate results}

To speed up testing, or to manipulate what is happening
internally, %\footnote{which isn't something for the fainthearted}
we can store and retrieve several important parts of our experiment:
The InstanceBase, the FeatureWeights, and the ProbabilityArrays.

Saving is done with:

\begin{verbatim}
  bool WriteInstanceBase( const char *f );
  bool SaveWeights( const char *f );
  bool WriteArrays( const char *f );
\end{verbatim}

Retrieve with their counterparts:

\begin{verbatim}
  bool GetInstanceBase( const char *f );
  bool GetWeights( const char *f );
  bool GetArrays( const char *f );
\end{verbatim}

All use f as a filename for storing/retrieving.

Some notes:\\
\begin{enumerate}
\item The InstanceBase is stored in a internal format, with or without
hashing, depending on the -H option. The format is described in the
Timbl manual. Remember that it is a bad idea to edit this file in any way.
\item  SaveWeights() only saves the current active weights, so you loose
information on the whereabouts, such as whether it were IG weights or
GR weights. GetWeights() can be used to override the weights that
Learn() calculated. 
\item The Probability arrays are described in the Timbl manual. They can be
manipulated to tune MVDM testing.
\end{enumerate}

\section{Do it ourselves}

After an experiment is trained with Learn(), we don't have to use
Test() to do bulk-testing on a file.
We can wrap our own tests with the Classify functions:

\begin{verbatim}
  bool Classify( const char *Line, const char *& result );
  bool Classify( const char *Line, const char *& result, 
                 double & distance );
  bool Classify( const char *Line, const char *& result,
                 char *& Distrib, double & distance );
\end{verbatim}

All versions will classify Line against the InstanceBase.\\
Results are stored in 'result' (the assigned class). 'distance' will
contain the calculated distance, and 'Distrib' the distribution at
'distance' which is used to calculate 'result'.

A main disadvantage compared to using Test() is, that Test() is optimized.
Classify() has to test for sanity of its input and also whether a
SetOptions() has been performed. This slows down the process of
course. \\
\ \\
A good example of the use of Classify() is the classify.cxx program in
the Timbl Distribution.

Other functions that can be used to create your own special effects:

\begin{verbatim}
 bool Increment( const char *Line ); 
 bool Decrement( const char *Line ); 
\end{verbatim}

These functions add or remove an Instance as described by Line to or
from the InstanceBase.
This can only be done for IB1-like experiments (IB1, IB2, CV and LOO).
This enforces a lot of statistical recalculations.

More sophisticated are:
\begin{verbatim}
 bool Expand( const char *File  );
 bool Remove( const char *File );
\end{verbatim}

which use the contents of File to do a bulk of Increments or Decrements, and
recalculate afterwards.

\section{Getting more information out of Timbl}

There are a few convenience functions to get extra information on
Timbl and its behaviour:

\begin{verbatim}
  bool WriteNamesFile( const char *f );
\end{verbatim}

Create a file which resembles a C4.5 namesfile.

\begin{verbatim}
  Algorithm Algo()
\end{verbatim}

Give the current algorithm as a type enum Algorithm.

Declaration of Algorithm:
\begin{verbatim}
enum Algorithm { UNKNOWN_ALG, IB1, IB2, IGTREE, 
                 TRIBL, TRIBL2, LOO, CV };
\end{verbatim}

\begin{verbatim}
  char *ExpName()
\end{verbatim}

Return the value of 'name' given at the construction of the experiment


\begin{verbatim}
  int Version()
  int Revision()
  char *RevCmnt()
\end{verbatim}

Give the Version number, the Revision and the Revision string of the
current API implementation.

\section{Using Timbl as a Server}

\begin{verbatim}
     bool StartServer( const int port );
\end{verbatim}

Start a TimblServer on 'port'. This makes sense only after the
experiment is trained.

\section{Appendix A}

\begin{tabular}{|r|r|l|}
\hline
\multicolumn{3}{|c|}{Options that only can be set {\bf once}}\\
\hline
option & value & description \\
\hline
-a &    & algorithm \\
   & 0 or IB1   & IB1     (default)\\
   & 1 or IG    & IGTree \\
   & 2 or TRIBL & TRIBL \\
   & 3 or IB2   & IB2 \\
   & 4 or TRIBL2 & TRIBL2 \\
-F & format & Assume the specified inputformat. \\
 & &     (Compact, C4.5, ARFF, Columns or Binary) \\
-l & integer & length of Features (Compact format only). \\
-M & integer & size of MaxBests Array \\
-q & integer & TRIBL treshold at level n. \\
-T &     & internal order of the Tree : \\
   & DO  & none. \\
   & GRO & using GainRatio \\
   & IGO & using InformationGain \\
   & 1/V & using 1/\# of Vals \\
   & G/V & using GainRatio/\# of Vals \\
   & I/V & using InfoGain/\# of Vals \\
   & X2O & using X-square \\
   & X/V & using X-square/\# of Vals \\
   & SVO & using Shared Variance \\
   & S/V & using Shared Variance/\# of Vals \\
   & 1/S & using 1/SplitInfo \\
\hline
\multicolumn{3}{|c|}{Other options}\ \\
\hline
-b & integer  & number of lines used for bootstrapping (IB2 only)\\
-B & integer  & number of bins used for discretization of numeric feature values\\
-d &     & weight neighbors as function of their distance: \\
   &  Z  & all the same weight. (default) \\
   & ID  & Inverse Distance. \\
   & IL  & Inverse Linear \\
   & EDa & Exponential Decay with factor a. (no whitespace!) \\
-e & n   &  estimate time until n patterns tested. \\
+H & & write hashed trees (default) \\
-H & & don't write hashed trees  \\
-k & integer & k nearest neighbors (default n = 1). \\
-m &   &  metrics \\
   & O & weighted overlap. (default) \\
   & M & modified value difference. \\
   & N & numeric values. \\ 
   & I & Ignore named  values. \\
-N & integer & Number of features \\
-p & integer & show progress every n lines. (default p = 100,000) \\
\hline
\end{tabular}

\newpage

\begin{tabular}{|l|r|l|}
\hline
option & value & description \\
\hline
-R & integer & solve ties at random with seed n. \\
% -R & P n & solve ties probalistic with seed n \\
-s &  & expect exemplar weights in all input files, and use for training \\
   & 0 & expect but ignore exemplar weights from all files \\
   & 1 & ignore exemplar weights from the test file \\
-t &  leave\_one\_out &  WERKT DIT?? \\
-t &  cross\_validate & WERKT DIT?? \\
+v && \\
-v &    & set or unset verbosity level, where level is \\
   & s  & work silently. \\
   & o  & show all options set. \\
   & f  & show Calculated Feature Weights. (default) \\
   & p  & show MVD matrices. \\
   & e  & show exact matches. \\
   & cm & show Confusion Matrix. \\
   & di & add distance to output file. \\
   & db & add distribution of best matched to output file \\
   & n  & add nearest neigbors to output file   (sets -x and --) \\
   &    &  You may combine levels using '+' e.g. +v p+db or -v o+di \\
-V & & Show VERSION. \\
-w &  & weighting \\
   & 0 or nw & No Weighting. \\
   & 1 or gr & Weight using GainRatio. (default) \\
   & 2 or ig & Weight using InfoGain \\
   & 3 or x2 & Weight using Chi-square \\
   & 4 or sv & Weight using Shared Variance \\
   & name & use Weights from file 'name' \\
+x & & use the exact match shortcut. (IB only) \\
-x & & don't use the exact match shortcut. (IB only, default) \\
+- & &  use inverted files. (IB1 only) \\
-- & &  don't use inverted files. (IB1 only, default, except for
Binary input) \\
\hline
\end{tabular}

\clearpage
\section{Appendix B: Annotated example programs}

\subsection{example 1, api\_test.cxx}
\begin{verbatim}	
#include <iosfwd>
#include "TimblAPI.h"

int main(){
  TimblAPI *My_Experiment = new TimblAPI( "-a IGTREE +vDI+DB" , 
                                          "test1" );
  My_Experiment->SetOptions( "-w2 -mM" );
  My_Experiment->SetOptions( "-w3 -vDB" );
  My_Experiment->ShowSettings( cout );
  My_Experiment->Learn( "dimin.train" );  
  My_Experiment->Test( "dimin.test", "my_first_test" );  
}

\end{verbatim}

Output:
\begin{verbatim}
Current Experiment Settings :
FLENGTH              : 0
MAXBESTS             : 500
TRIBL_OFFSET         : 0
INPUTFORMAT          : Unknown
TREE_ORDER           : Unknown
DECAY                : Z
SEED                 : -1
DECAYPARAM           : 1.000000
EXEMPLAR_WEIGHTS     : -
IGNORE_EXEMPLAR_WEIGHT : +
NO_EXEMPLAR_WEIGHTS_TEST : +
PROBALISTIC          : -
VERBOSITY            : F+DI                        [Note 1]
EXACT_MATCH          : -
USE_INVERTED         : -
HASHED_TREES         : +
GLOBAL_METRIC        : M                           [Note 2]
METRICS              : 
NEIGHBORS            : 1
PROGRESS             : 100000
IB2_OFFSET           : 0
BIN_SIZE             : 20
WEIGHTING            : x2                          [Note 3]

Examine datafile gave the following results:
Number of Features: 12
InputFormat       : C4.5

-test1-Phase 1: Reading Datafile: dimin.train
-test1-Start:          0 @ Mon Apr 22 16:56:43 2002
-test1-Finished:    2999 @ Mon Apr 22 16:56:43 2002
-test1-Calculating Entropy         Mon Apr 22 16:56:43 2002
Lines of data     : 2999
DB Entropy        : 1.6178929
Number of Classes : 5

Feats  Vals  X-square   Variance     InfoGain     GainRatio
  1      3   128.41828  0.021410184  0.030971064  0.024891536
  2     50   364.75812  0.030406645  0.060860038  0.027552191
  3     19   212.29804  0.017697402  0.039562857  0.018676787
  4     37   449.83823  0.037499019  0.052541227  0.052620750
  5      3   288.87218  0.048161417  0.074523225  0.047699231
  6     61   415.64113  0.034648310  0.10604433   0.024471911
  7     20   501.33465  0.041791818  0.12348668   0.034953203
  8     69   367.66021  0.030648567  0.097198760  0.043983864
  9      2   169.36962  0.056475363  0.045752381  0.046816705
 10     64   914.61906  0.076243669  0.21388759   0.042844587
 11     18   2807.0418  0.23399815   0.66970458   0.18507018
 12     43   7160.3682  0.59689631   1.2780762    0.32537181

Feature Permutation based on Chi-Squared :
< 12, 11, 10, 7, 4, 6, 8, 2, 5, 3, 9, 1 >
-test1-Phase 2: Learning from Datafile: dimin.train
-test1-Start:          0 @ Mon Apr 22 16:56:43 2002
-test1-Finished:    2999 @ Mon Apr 22 16:56:44 2002

Size of InstanceBase = 29481 Nodes, (589620 bytes), 23.00 % compression

-test1-Start Pruning:         Mon Apr 22 16:56:44 2002
-test1-Finished Pruning:         Mon Apr 22 16:56:44 2002

Size of InstanceBase = 148 Nodes, (2960 bytes), 99.61 % compression

Warning:-test1-Metric set to Overlap for IGTree test.           [Note 2]
Examine datafile gave the following results:
Number of Features: 12
InputFormat       : C4.5


Starting to test, Testfile: dimin.test
Writing output in:          my_first_test
Algorithm     : IGTree

-test1-Tested:      1 @ Mon Apr 22 16:56:44 2002
-test1-Tested:      2 @ Mon Apr 22 16:56:44 2002
-test1-Tested:      3 @ Mon Apr 22 16:56:44 2002
-test1-Tested:      4 @ Mon Apr 22 16:56:44 2002
-test1-Tested:      5 @ Mon Apr 22 16:56:44 2002
-test1-Tested:      6 @ Mon Apr 22 16:56:44 2002
-test1-Tested:      7 @ Mon Apr 22 16:56:44 2002
-test1-Tested:      8 @ Mon Apr 22 16:56:44 2002
-test1-Tested:      9 @ Mon Apr 22 16:56:44 2002
-test1-Tested:     10 @ Mon Apr 22 16:56:44 2002
-test1-Tested:    100 @ Mon Apr 22 16:56:44 2002
-test1-Ready:     950 @ Mon Apr 22 16:56:45 2002
Seconds taken: 1 (950.00 p/s)
914/950 (0.962105)

\end{verbatim}

Notes:
\begin{enumerate}
\item The first SetOption() sets the verbosity with +vDI+DV.
The default for v is F so the verbosity should become F+DI+DB
The second SetOption() however sets the verbosity with -vDB, and the
resulting verbosity is therefore F+DI
\item Due to the second SetOption(), the Default metric is set to
MVDM, this is however not applicable to IGTREE. This raises a warning
later on, when we start to test and the API informs us that Overlap is
used instead.
\item The -w2 of the first SetOption() is overruled with -w3 from the
second SetOption(), resulting in a weighting of 3 or Chi-Square.
\end{enumerate}

Result in my\_first\_test (first 20 lines):
\begin{verbatim}
=,=,=,=,=,=,=,=,+,p,e,=,T,T        6619.851263
=,=,=,=,+,k,u,=,-,bl,u,m,E,P        2396.855798
+,m,I,=,-,d,A,G,-,d,},t,J,J        6619.851263
-,t,@,=,-,l,|,=,-,G,@,n,T,T        6619.851263
-,=,I,n,-,str,y,=,+,m,E,nt,J,J        6619.851263
=,=,=,=,=,=,=,=,+,br,L,t,J,J        6619.851263
=,=,=,=,+,zw,A,=,-,m,@,r,T,T        6619.851263
=,=,=,=,-,f,u,=,+,dr,a,l,T,T        6619.851263
=,=,=,=,=,=,=,=,+,l,e,w,T,T        13780.219415
=,=,=,=,+,tr,K,N,-,k,a,rt,J,J        6619.851263
=,=,=,=,+,=,o,=,-,p,u,=,T,T        3812.809510
=,=,=,=,=,=,=,=,+,l,A,m,E,E        3812.809510
=,=,=,=,=,=,=,=,+,l,A,p,J,J        6619.851263
=,=,=,=,=,=,=,=,+,sx,E,lm,P,P        6619.851263
+,l,a,=,-,d,@,=,-,k,A,st,J,J        6619.851263
-,s,i,=,-,f,E,r,-,st,O,k,J,J        6619.851263
=,=,=,=,=,=,=,=,+,sp,a,n,T,T        6619.851263
=,=,=,=,=,=,=,=,+,st,o,t,J,J        6619.851263
=,=,=,=,+,sp,a,r,-,b,u,k,J,J        6619.851263
+,h,I,N,-,k,@,l,-,bl,O,k,J,J        6619.851263
-,m,e,=,-,d,A,l,+,j,O,n,E,E        3812.809510
\end{verbatim}

\subsection{example 2, api\_test2.cxx}
This demonstrates IB2 learning.

Our example program:

\begin{verbatim}
#include <iostream>
#include "TimblAPI.h"

int main(){
  TimblAPI *My_Experiment = new TimblAPI( "-a IB2 +vDI+DB" , 
                                          "test2" );
  My_Experiment->SetOptions( "-b100" );
  My_Experiment->ShowSettings( std::cout );
  My_Experiment->Learn( "dimin.train" );  
  My_Experiment->Test( "dimin.test", "my_second_test.out" );  
  exit(0);
}
\end{verbatim}

We create an experiment for the IB2 algorithm, with the 'b' option set
to 100, so the first 100 lines of ``dimin.train'' will be used to
Bootstrap the learning, as we can see from the output:

\begin{verbatim}
Current Experiment Settings :
FLENGTH              : 0
MAXBESTS             : 500
TRIBL_OFFSET         : 0
INPUTFORMAT          : Unknown
TREE_ORDER           : G/V
DECAY                : Z
SEED                 : -1
DECAYPARAM           : 1.000000
EXEMPLAR_WEIGHTS     : -
IGNORE_EXEMPLAR_WEIGHT : +
NO_EXEMPLAR_WEIGHTS_TEST : +
PROBALISTIC          : -
VERBOSITY            : F+DI+DB
EXACT_MATCH          : -
USE_INVERTED         : -
HASHED_TREES         : +
GLOBAL_METRIC        : O
METRICS              : 
NEIGHBORS            : 1
PROGRESS             : 100000
IB2_OFFSET           : 100
BIN_SIZE             : 20
WEIGHTING            : gr

Examine datafile gave the following results:
Number of Features: 12
InputFormat       : C4.5

-test2-Phase 1: Reading Datafile: dimin.train
-test2-Start:          0 @ Mon Jun 10 14:07:57 2002
-test2-Finished:     100 @ Mon Jun 10 14:07:57 2002
-test2-Calculating Entropy         Mon Jun 10 14:07:57 2002
Lines of data     : 100
DB Entropy        : 1.7148291
Number of Classes : 5

Feats   Vals    X-square    Variance     InfoGain     GainRatio
    1      3    7.6024645   0.038012322  0.059969698  0.041607175
    2     16    46.203035   0.11550759   0.31505423   0.14385015
    3     13    54.694253   0.13673563   0.28508033   0.11717175
    4     12    82.839834   0.20709958   0.29549424   0.20791549
    5      3    9.8142041   0.049071021  0.081665925  0.053303154
    6     25    93.713133   0.23428283   0.52671728   0.13537479
    7     15    56.143669   0.14035917   0.35043203   0.10312857
    8     21    49.386590   0.12346647   0.33767640   0.14930809
    9      2    3.1042762   0.031042762  0.031832973  0.033768474
   10     29    148.77401   0.37193502   0.70753849   0.15973455
   11     14    136.42479   0.34106198   0.90357639   0.25337382
   12     22    219.58669   0.54896672   1.3841139    0.34375881

Feature Permutation based on GainRatio/Values :
< 11, 5, 4, 9, 12, 1, 3, 2, 8, 7, 10, 6 >
-test2-Phase 2: Learning from Datafile: dimin.train
-test2-Start:          0 @ Mon Jun 10 14:07:57 2002
-test2-Finished:     100 @ Mon Jun 10 14:07:57 2002

Size of InstanceBase = 991 Nodes, (19820 bytes), 23.77 % compression

-test2-Phase 2: Appending from Datafile: dimin.train (starting at line 101)
-test2-Start:        101 @ Mon Jun 10 14:07:57 2002
-test2-Finished:    2999 @ Mon Jun 10 14:07:58 2002
added 242 new entries                                      [Note 1]

Size of InstanceBase = 2807 Nodes, (56140 bytes), 32.52 % compression

DB Entropy        : 2.06260672
Number of Classes : 5

Feats   Vals   X-square      Variance     InfoGain     GainRatio
    1      3   19.21211054   0.02808788   0.04663668   0.03472093
    2     26   120.99000006  0.08844298   0.21552161   0.09862660
    3     15   95.37436002   0.06971810   0.17130436   0.07700194
    4     16   87.51574055   0.06397349   0.13415866   0.11868667
    5      3   31.00890345   0.04533465   0.07709044   0.04882750
    6     32   142.15589806  0.10391513   0.30555295   0.07911265
    7     18   139.20481395  0.10175790   0.26973542   0.08142467
    8     30   97.89853223   0.07156325   0.18433394   0.09439585
    9      2   26.57144438   0.07769428   0.07776679   0.07804426
   10     42   234.33268308  0.17129582   0.45128191   0.09418234
   11     16   286.34676422  0.20931781   0.63347440   0.17490482
   12     40   751.89378418  0.54962996   1.35537142   0.31983281

Examine datafile gave the following results:
Number of Features: 12
InputFormat       : C4.5


Starting to test, Testfile: dimin.test
Writing output in:          my_second_test.out
Algorithm     : IB2
Global metric : Overlap
Deviant Feature Metrics:(none)
Weighting     : GainRatio
Feature 1	 : 0.03472093
Feature 2	 : 0.09862660
Feature 3	 : 0.07700194
Feature 4	 : 0.11868667
Feature 5	 : 0.04882750
Feature 6	 : 0.07911265
Feature 7	 : 0.08142467
Feature 8	 : 0.09439585
Feature 9	 : 0.07804426
Feature 10	 : 0.09418234
Feature 11	 : 0.17490482
Feature 12	 : 0.31983281

-test2-Tested:      1 @ Mon Jun 10 14:07:58 2002
-test2-Tested:      2 @ Mon Jun 10 14:07:58 2002
-test2-Tested:      3 @ Mon Jun 10 14:07:58 2002
-test2-Tested:      4 @ Mon Jun 10 14:07:58 2002
-test2-Tested:      5 @ Mon Jun 10 14:07:58 2002
-test2-Tested:      6 @ Mon Jun 10 14:07:58 2002
-test2-Tested:      7 @ Mon Jun 10 14:07:58 2002
-test2-Tested:      8 @ Mon Jun 10 14:07:58 2002
-test2-Tested:      9 @ Mon Jun 10 14:07:58 2002
-test2-Tested:     10 @ Mon Jun 10 14:07:58 2002
-test2-Tested:    100 @ Mon Jun 10 14:07:58 2002
-test2-Ready:     950 @ Mon Jun 10 14:07:59 2002
Seconds taken: 1 (950.00 p/s)
889/950 (0.935789), of which 15 exact matches       [Note 2]

There were 57 ties of which 48 (84.21%) were correctly resolved
\end{verbatim}

Notes:
\begin{enumerate}
\item As we see here, 242 entries from the inputfile had a mismatch,
and were therefore entered in the Instancebase.
\item We see that IB2 scores 93.58 \%, compared to 96.21 \% for IGtree
in our first example. 
For this data, IB2 is not a good algorithm. However, it saves a
lot of space, and is faster than IB1. However, IGtree is both faster and
better. Had we used IB1, the score would have been 96.84 \%.
\end{enumerate}

\subsection{example 3, api\_test3.cxx}
This demonstrates Cross Validation:

Lets try the following program:

\begin{verbatim}
#include "TimblAPI.h"
int main(){
  TimblAPI *My_Experiment = new TimblAPI( "-t cross_validate" );
  My_Experiment->Test( "cross_val.test" );  
}
\end{verbatim}

We create an experiment, which defaults to IB1 and because of the
special option ``-t cross\_validate'' will start a CrossValidation
experiment.\\
Learn() is not possible now. We must use a special form of Test().

``cross\_val.test'' is a file with the following content:
\begin{verbatim}
klein_1.train
klein_2.train
klein_3.train
klein_4.train
klein_5.train
\end{verbatim}

All these files contain an equal part of a bigger dataset, and
My\_Experiment will run a CrossValidation test between these files.
Note that output filenames are generated and that you can't influence
that.\\

The output of this program is:

\begin{verbatim}
Starting Cross validation test on files:
klein_1.train
klein_2.train
klein_3.train
klein_4.train
klein_5.train
Examine datafile gave the following results:
Number of Features: 8
InputFormat       : C4.5

DB Entropy        : 1.1239577
Number of Classes : 3

Feats  Vals  X-square   Variance    InfoGain    GainRatio
  1     31   73.159596  0.96262626  1.0514606   0.21636590
  2     31   76.000000  1.0000000   1.1239577   0.23128409
  3     29   51.875758  0.68257576  0.96606292  0.20546896
  4     30   38.594949  0.50782828  0.76843726  0.16052103
  5     17   43.164545  0.56795455  0.75290857  0.19555822
  6     17   76.000000  1.0000000   1.1239577   0.29193340
  7     18   39.410606  0.51856061  0.86079976  0.21819805
  8     17   21.504545  0.28295455  0.40001621  0.10329552


Starting to test, Testfile: klein_1.train
Writing output in:          klein_1.train.cv
Algorithm     : CV
Global metric : Overlap
Deviant Feature Metrics:(none)
Weighting     : GainRatio
Feature 1	 : 0.21636590
Feature 2	 : 0.23128409
Feature 3	 : 0.20546896
Feature 4	 : 0.16052103
Feature 5	 : 0.19555822
Feature 6	 : 0.29193340
Feature 7	 : 0.21819805
Feature 8	 : 0.10329552

Tested:      1 @ Tue Apr 23 14:39:13 2002
Tested:      2 @ Tue Apr 23 14:39:13 2002
Tested:      3 @ Tue Apr 23 14:39:13 2002
Tested:      4 @ Tue Apr 23 14:39:13 2002
Tested:      5 @ Tue Apr 23 14:39:13 2002
Tested:      6 @ Tue Apr 23 14:39:13 2002
Tested:      7 @ Tue Apr 23 14:39:13 2002
Tested:      8 @ Tue Apr 23 14:39:13 2002
Tested:      9 @ Tue Apr 23 14:39:13 2002
Tested:     10 @ Tue Apr 23 14:39:13 2002
Ready:      10 @ Tue Apr 23 14:39:13 2002
Seconds taken: 1 (10.00 p/s)
8/10 (0.800000)
Examine datafile gave the following results:
Number of Features: 8
InputFormat       : C4.5

DB Entropy        : 1.21081003
Number of Classes : 3

Feats  Vals  X-square     Variance    InfoGain    GainRatio
  1     30   70.81818182  0.93181818  1.08568141  0.22585381
  2     29   73.03896104  0.96103896  1.13831299  0.24042877
  3     25   67.85714286  0.89285714  1.01318437  0.23032898
  4     27   54.45021645  0.71645022  0.93041471  0.20656967
  5     15   35.69696970  0.46969697  0.75002647  0.20299617
  6     15   73.03896104  0.96103896  1.13831299  0.30537853
  7     16   39.89177489  0.52489177  0.83542417  0.22216874
  8     15   21.70253556  0.28555968  0.44603650  0.12394414


Starting to test, Testfile: klein_2.train
Writing output in:          klein_2.train.cv
Algorithm     : CV
Global metric : Overlap
Deviant Feature Metrics:(none)
Weighting     : GainRatio
Feature 1	 : 0.22585381
Feature 2	 : 0.24042877
Feature 3	 : 0.23032898
Feature 4	 : 0.20656967
Feature 5	 : 0.20299617
Feature 6	 : 0.30537853
Feature 7	 : 0.22216874
Feature 8	 : 0.12394414

Tested:      1 @ Tue Apr 23 14:39:13 2002
Tested:      2 @ Tue Apr 23 14:39:13 2002
Tested:      3 @ Tue Apr 23 14:39:13 2002
Tested:      4 @ Tue Apr 23 14:39:13 2002
Tested:      5 @ Tue Apr 23 14:39:13 2002
Tested:      6 @ Tue Apr 23 14:39:13 2002
Tested:      7 @ Tue Apr 23 14:39:13 2002
Tested:      8 @ Tue Apr 23 14:39:13 2002
Tested:      9 @ Tue Apr 23 14:39:13 2002
Tested:     10 @ Tue Apr 23 14:39:13 2002
Ready:      10 @ Tue Apr 23 14:39:13 2002
Seconds taken: 1 (10.00 p/s)
8/10 (0.800000)
Examine datafile gave the following results:
Number of Features: 8
InputFormat       : C4.5

DB Entropy        : 1.10727241
Number of Classes : 3

Feats  Vals  X-square     Variance    InfoGain    GainRatio
  1     33   73.81677019  0.97127329  1.05464083  0.21157263
  2     31   73.81677019  0.97127329  1.05464083  0.21613678
  3     29   70.54192547  0.92818323  0.96924313  0.20527803
  4     29   63.77391304  0.83913043  0.80175556  0.17134265
  5     17   32.59109731  0.42883023  0.73188656  0.19159785
  6     17   73.81677019  0.97127329  1.05464083  0.27609049
  7     17   33.44387755  0.44005102  0.76788828  0.20441129
  8     16   15.45766046  0.20339027  0.30124108  0.08102333


Starting to test, Testfile: klein_3.train
Writing output in:          klein_3.train.cv
Algorithm     : CV
Global metric : Overlap
Deviant Feature Metrics:(none)
Weighting     : GainRatio
Feature 1	 : 0.21157263
Feature 2	 : 0.21613678
Feature 3	 : 0.20527803
Feature 4	 : 0.17134265
Feature 5	 : 0.19159785
Feature 6	 : 0.27609049
Feature 7	 : 0.20441129
Feature 8	 : 0.08102333

Tested:      1 @ Tue Apr 23 14:39:13 2002
Tested:      2 @ Tue Apr 23 14:39:13 2002
Tested:      3 @ Tue Apr 23 14:39:13 2002
Tested:      4 @ Tue Apr 23 14:39:13 2002
Tested:      5 @ Tue Apr 23 14:39:13 2002
Tested:      6 @ Tue Apr 23 14:39:13 2002
Tested:      7 @ Tue Apr 23 14:39:13 2002
Tested:      8 @ Tue Apr 23 14:39:13 2002
Tested:      9 @ Tue Apr 23 14:39:13 2002
Tested:     10 @ Tue Apr 23 14:39:13 2002
Ready:      10 @ Tue Apr 23 14:39:13 2002
Seconds taken: 1 (10.00 p/s)
9/10 (0.900000)
Examine datafile gave the following results:
Number of Features: 8
InputFormat       : C4.5

DB Entropy        : 1.16743672
Number of Classes : 3

Feats  Vals  X-square     Variance    InfoGain    GainRatio
  1     32   73.62500000  0.96875000  1.11480514  0.22602878
  2     32   72.83333333  0.95833333  1.09493968  0.22289880
  3     30   59.37500000  0.78125000  0.96981107  0.20714134
  4     31   63.33333333  0.83333333  1.06217357  0.22200231
  5     16   43.54166667  0.57291667  0.71361797  0.19418620
  6     16   72.83333333  0.95833333  1.09493968  0.29899921
  7     17   37.09523810  0.48809524  0.76845621  0.20564998
  8     15   24.78670635  0.32614087  0.53212858  0.14797537


Starting to test, Testfile: klein_4.train
Writing output in:          klein_4.train.cv
Algorithm     : CV
Global metric : Overlap
Deviant Feature Metrics:(none)
Weighting     : GainRatio
Feature 1	 : 0.22602878
Feature 2	 : 0.22289880
Feature 3	 : 0.20714134
Feature 4	 : 0.22200231
Feature 5	 : 0.19418620
Feature 6	 : 0.29899921
Feature 7	 : 0.20564998
Feature 8	 : 0.14797537

Tested:      1 @ Tue Apr 23 14:39:13 2002
Tested:      2 @ Tue Apr 23 14:39:13 2002
Tested:      3 @ Tue Apr 23 14:39:13 2002
Tested:      4 @ Tue Apr 23 14:39:13 2002
Tested:      5 @ Tue Apr 23 14:39:13 2002
Tested:      6 @ Tue Apr 23 14:39:13 2002
Tested:      7 @ Tue Apr 23 14:39:13 2002
Tested:      8 @ Tue Apr 23 14:39:13 2002
Tested:      9 @ Tue Apr 23 14:39:13 2002
Tested:     10 @ Tue Apr 23 14:39:13 2002
Ready:      10 @ Tue Apr 23 14:39:13 2002
Seconds taken: 1 (10.00 p/s)
8/10 (0.800000)
Examine datafile gave the following results:
Number of Features: 8
InputFormat       : C4.5

DB Entropy        : 1.16687507
Number of Classes : 3

Feats  Vals  X-square     Variance    InfoGain    GainRatio
  1     32   75.32307692  0.94153846  1.06687507  0.21675958
  2     31   77.66153846  0.97076923  1.11687507  0.22924703
  3     29   63.35384615  0.79192308  0.98574726  0.20875948
  4     28   55.36615385  0.69207692  0.82663406  0.17847364
  5     16   35.75641026  0.44695513  0.69800288  0.18631101
  6     16   77.66153846  0.97076923  1.11687507  0.29662222
  7     16   38.67692308  0.48346154  0.78574726  0.21148893
  8     16   19.69692308  0.24621154  0.37663406  0.10195991


Starting to test, Testfile: klein_5.train
Writing output in:          klein_5.train.cv
Algorithm     : CV
Global metric : Overlap
Deviant Feature Metrics:(none)
Weighting     : GainRatio
Feature 1	 : 0.21675958
Feature 2	 : 0.22924703
Feature 3	 : 0.20875948
Feature 4	 : 0.17847364
Feature 5	 : 0.18631101
Feature 6	 : 0.29662222
Feature 7	 : 0.21148893
Feature 8	 : 0.10195991

Tested:      1 @ Tue Apr 23 14:39:13 2002
Tested:      2 @ Tue Apr 23 14:39:13 2002
Tested:      3 @ Tue Apr 23 14:39:13 2002
Tested:      4 @ Tue Apr 23 14:39:13 2002
Tested:      5 @ Tue Apr 23 14:39:13 2002
Tested:      6 @ Tue Apr 23 14:39:13 2002
Tested:      7 @ Tue Apr 23 14:39:13 2002
Tested:      8 @ Tue Apr 23 14:39:13 2002
Ready:       8 @ Tue Apr 23 14:39:13 2002
Seconds taken: 1 (8.00 p/s)
8/8 (1.000000)
\end{verbatim}

What has happened here?

\begin{enumerate}
\item Timbl trained itself with inputfiles klein\_2.train through
klein\_5.train. (in fact using the {\tt Expand()} API call.
\item Then Timbl tested klein\_1.train against the InstanceBase.
\item next, klein\_2.train is removed from the database (API call {\tt
Remove()} ) and klein\_1.train is added.
\item Then klein\_2.train is tested against the InstanceBase
\item and so forth with klein\_3.train ...
\end{enumerate}

\end{document}
