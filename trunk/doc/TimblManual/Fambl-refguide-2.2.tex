% FAMBL 2.2 reference guide
% copyright 1997-2004 ILK / Tilburg University
% Antal van den Bosch / Antal.vdnBosch@uvt.nl

\documentclass[11pt]{article}
\usepackage{epsf}
\usepackage{a4wide}
\usepackage{palatino}
\usepackage[english]{babel}

\parindent 0pt
\parskip 9pt

\begin{document}

\title{FAMBL 2.2 \\ {\sl Reference Guide}}
\author{Antal van den Bosch \\ ILK / Computational Linguistics and AI\\
Tilburg University, The Netherlands \\ {\tt http://ilk.uvt.nl} }

\maketitle

\vspace*{6cm}

\rule{\textwidth}{1.0mm}

{\sc fambl}, Family-Based Learner, version 2.2, is a freely available
software package for research or educational purposes only. 

{\sc fambl} comes WITHOUT ANY WARRANTY. Author nor distributor accept
responsibility to anyone for the consequences of using it or for
whether it serves any particular purpose or works at all.

\copyright 1997--2004 ILK / Tilburg University.

\rule{\textwidth}{0.5mm}

\clearpage

\section{FAMBL: Family-Based Learning}

This document is intended as a quick reference guide of the {\sc
fambl} algorithm and its current implementation, version 2.2. {\sc
fambl} is a memory-based learning algorithm that carefully generalizes
over instances. It merges nearest-neighbour instances (exemplars) of
the same class into generalized exemplars. In {\sc fambl}, these are
called families. {\sc fambl} combines earlier approaches to
generalized exemplars, {\sc nge} (Salzberg, 1991) and {\sc rise}
(Domingos, 1995), with some new ideas and heuristics.

{\sc fambl} is documented in two papers (Van den Bosch, 1999a, 1999b)
that come with the {\sc fambl} software distribution (see next
section). The papers describe the core element, generalizing instances
into families, that makes {\sc fambl} different from pure memory-based
learning. The latter is implemented in the TiMBL software
package\footnote{The current version, TiMBL main version 5, can be
  downloaded from {\tt http://ilk.uvt.nl}.}. {\sc fambl} should be
seen as a specialised variant of TiMBL. If you are interested in
performing pure memory-based learning without generalized instances or
families, TiMBL is the preferred package due to its faster learning
speed. If you do wish to use {\sc fambl} for experimentation, you have
to install TiMBL: {\sc fambl} makes use of some of TiMBL's efficient
procedures in preprocessing data.

{\sc Fambl} is a special variant of memory-based learning that has the
ability to generalise over instances. {\sc Fambl} devotes a special
learning phase to converting the original instance base (the starting
point for standard memory-based classification) into a set of {\sl
families}. Families are merged instances that are closely alike, and
that are associated with the same classification. Merging takes the
form of making disjunctions of values for those symbolic feature
values on which family members mismatch. For numeric values, merging
generates sides of hyperrectangles that span between the minimum and
maximum values of the merged features. After families are extracted
from the original instance base, the latter is discarded and all
further classification (e.g. of test material) is based on
nearest-neighbour matching with the families. Depending on the type of
data, this may allow for considerable memory compression, and
sometimes improvements in classification accuracy and speed.

The current implementation of {\sc fambl} is equipped with various
optional metrics for feature weighting (including weighting of feature
values and feature combinations), symbolic value difference
estimation, and exemplar weighting -- more than are mentioned and
tested in the original papers. This document is intended to
describe all options available via command-line arguments. These
descriptions tend to refer to background aspects of the {\sc fambl}
algorithm, but an attempt is made to make the descriptions generally
understandable.

The structure of the document is as follows. First, in
Section~\ref{install}, the install procedure of the {\sc fambl}
software is outlined. Section~\ref{changes} describes the changes made
since previous public releases of {\sc fambl}. In
Section~\ref{command}, a list of command-line options is given, with a
description of each option. The data format is described in
Section~\ref{format}. Section~\ref{examples} provides an annotated
example of a command line and its effects when {\sc fambl} is run with
it.  Section~\ref{notes} notes some limitations and problems of the
current program.

\section{Installing the software}
\label{install}

{\sc Fambl} 2.2 is written in fairly ANSI C.  It should compile and
run on most UNIX systems. It can be downloaded from

\begin{quote}
	http://ilk.uvt.nl/downloads/pub/software/Fambl.2.2.tar.gz
\end{quote}

After downloading this gzipped tar file, do the following (``$>$'' is
your shell promt):

\begin{enumerate}
\item
$>$ tar zxvf Fambl.2.2.tar.gz
\item
$>$ cd Fambl2
\item
$>$ make
\end{enumerate}

The unpack operation (tar zxvf) makes the directory Fambl2, and
installs all source code in that directory. A ``make'' in that
directory makes a ``Fambl'' executable. Edit the Makefile if you want
to change compiler (gcc by default) or optimisation (-O3 by
default). Type `make clean' to remove all make output.

The distribution also includes three PDF files in the {\tt doc}
directory, {\bf Fambl-refguide.pdf} (this document), {\bf
  Fambl-jetai.pdf} (long paper), and {\bf Fambl-icml99.pdf} (short
paper), the two aforementioned background papers which outline the
learning and classification procedures of {\sc fambl}.

In the {\tt demo} directory, two files are included that serve as
example data. In Section~\ref{examples} we exemplify how {\sc fambl}
is run on this data.

\section{Changes}
\label{changes}

Changes between 2.2 and 2.0 include the following:

\begin{itemize}
\item
The family extraction phase has been reverted to a simple randomized
selection of initial instances. {\sc Fambl} is not dependent on TiMBL nor
does it compute class prediction strength in another way.
\item
Along overlap and {\sc mvdm}, Jeffrey Divergence has been added as a
similarity function. Jeffrey Divergence is a symmetric variant of
Kullback--Leibler divergence.
\item
Many internal code optimizations ensure that {\sc fambl} operates
considerably faster, especially with the {\sc mvdm} and Jeffrey Divergence
functions, as compared to previous versions.
\end{itemize}

Changes from 1.01 (the first public releese) to version 2.0 were the
following:

\begin{itemize}
\item
Most prominently, the two-stage learning procedure (consisting of a
probing phase and a family extraction phase in version 1.01) has changed:
        \begin{enumerate}
        \item
        First, the TiMBL software is called to learn the selected
        training set, and do a classification on the training
        set itself with $k=2$. On the basis of the output of this
        experiment, the {\em class prediction strength}\/ of each
        training instance is computed. This is
        done with Laplace correction: $cps_{i} = 
        \frac{(N_{i}-Ndif_{i})+1}{N_{i}+c}$, where $N_{i}$ is the
        number of occurrences that instance $i$ is a nearest neighbor
        of another instance or itself (the latter always happens once),
        and $Ndif_{i}$ is the number of occurrences in which $i$ was
        a nearest neighbor while having a different class as compared
        to its neighbor, and $c$ is the number of classes.
        \item
        Then, all instances are placed in reverse CPS ordering. Starting
        with the instance with the highest CPS, instances are picked
        one by one to be the starting point of a new family. Again,
        family limits are set to $k=2$, so as to not overestimate the
        CPS scores obtained in the first stage. Essentially this is a way
        to keep {\sc fambl} ``careful'' in the sense of the earlier
        version also described in the papers.
        \end{enumerate}
\item
The treatment of atomic feature values as individual, weightable
features has been fully implemented and debugged, where this was only
an experimental option in version 1.01. As explained in more detail in
Section~\ref{command}, it is now possible to declare all feature
values in a multi-valued instance base as binary. Additionally, it is
possible to look for combinations of feature values that have a weight
higher than the sum of the individual weights of its elements, up to a
user-defined cardinality.
\end{itemize}


\section{Command-line options}
\label{command}

For each option, a table displays (i) the option letter, (ii) its
optional arguments, usually integers within a range, of which
one is the default value, and (iii) a short description. Underneath each
table, a longer description of the option and its values is given.

The {\bf -f} option is obligatory, and has one argument (the name of
the file containing labeled training examples). All other options can
be set by the user, and if they are not set, the default value of that
argument is selected.  \ \\


\begin{tabular}{|p{0.1\textwidth}|p{0.3\textwidth}|p{0.52\textwidth}|}
\hline
{\tt -f} & {\sl filename} & specify the name of the file containing training examples \\
\hline
\end{tabular}

With {\bf -f filename} the file containing the learning material is
declared. This file must be present. \ \\

\begin{tabular}{|p{0.1\textwidth}|p{0.3\textwidth}|p{0.52\textwidth}|}
\hline
{\tt -t} & {\sl filename} & evaluate {\sc fambl} on unseen cases in file {\sl filename} \\
\hline
\end{tabular}

Classification performance of {\sc fambl} after family extraction is
evaluated by classifying all instances in {\sl filename}. When a test
file is not specified, the learning data (set with {\bf -t}) is used
as test material.  \ \\

\begin{tabular}{|p{0.1\textwidth}|p{0.3\textwidth}|p{0.52\textwidth}|}
\hline
{\tt -g} & [0,5] (default 2) & set feature weighting metric \\
\hline
\end{tabular}

{\bf -g} sets the feature weighting metric $w_{i}$ used in the basic
$k$-NN distance function, which computes the distance between the
instance or family$X$ and the instance $Y$, $\Delta(X,Y) =
\sum_{i=1}^{n}\ w_{i}\ \delta(x_{i},y_{i})$, where $n$ is the number
of features, and $x_{i}$ is the value of the $i$th feature of instance
$X$. {\sc fambl} currently hosts six options:

\begin{description}

\item[-g 0] All features have the same weight $w_{i}=1$.

\item[-g 1] {\em Information gain weighting}. The information gain of
feature $f$ is the difference in instance-base entropy (scrambledness
of information) between the situations without and with knowledge of
the value of that feature: $w_{f} = H(C) - \sum_{v \in V_{f}} P(v)
\times H(C|v)$, where $C$ is the set of class labels, $V_{f}$ is the
set of values for feature $f$, and $H(C) = - \sum_{c \in C} P(c)
\log_{2} P(c)$ is the entropy of the class labels. The probabilities
are estimated from relative frequencies in the training set.

\item[-g 2 (default)] {\em (Information) gain ratio weighting}
(Quinlan, 1993). The gain ratio of a feature $f$ is its information
gain divided by its split info, $si(f)= - \sum_{v \in V_{f}} P(v)
\log_{2} P(v)$, the entropy of its values. Dividing information gain
by split info avoids a bias in favor of features with more values.
\item[-g 3] {\em Chi-square weighting}, giving for each matrix between
feature values and classes a global number $\chi^{2}$ for the level of
surprise within the matrix: the result is the sum of squares of all
differences between the expected values in each cell (based on
averaging over rows and columns) with the observed values. See any
basic statistics book.

\item[-g 4] {\em Shared variance weighting}, derived from the
chi-square value $\chi$ of the matrix between feature values and
classes. Shared variance of a feature expresses in a percentage how
much that feature explains the classification of all data. Shared
variance is $\frac{\chi}{N (L - 1)}$ where $N$ is the number of
instances, and $L$ is the lesser of the number of feature values or
the number of classes.

\item[-g 5] {\em Log-likelihood weighting} expresses, not unlike
chi-square and shared variance, the deviation of the cooccurrence of
two events between the virtual situation in which they are completely
independent, and the actual situation in which they may be strongly
dependent. Unlike chi-square and shared variance, log-likelihood goes
for two observations only (in this case, one value and one
class). Therefore log-likelihood is used cumulatively here; i.e., the
result per feature is an accumulation of log-likelihoods between all
values and classes. The maximum value is normalised to $1.0$. For a
detailed treatment of log-likelihood, including an example
computation, see Dunning (1993).
\end{description}
\ \\

\begin{tabular}{|p{0.1\textwidth}|p{0.3\textwidth}|p{0.52\textwidth}|}
\hline
{\tt -m} & [0-2] (default 0) & set value difference metric \\
\hline
\end{tabular}

{\bf -m} sets the value-difference function $\delta(x_{i},y_{i})$, the
distance between the values of feature $i$ in instances $X$ and $Y$,
in the $\Delta(X,Y)$ function that sums these distances over all
features (cf. {\bf -g}). Three options are available:

\begin{description}

\item[-m 0 (default value)] {\em Overlap metric} (also known as
  Hamming distance, Manhattan metric, or L1 metric). For nominal data,
  $\delta(x_{i}, y_{i})$ simply returns a distance of $1$ with a
  mismatch, and a $0$ with a match. With numeric values, the
  difference is scaled by the minimal and maximal values found for
  that feature. In other words, \\
\begin{math}
\delta(x_{i}, y_{i}) = \left\{ \begin{array}{ll}
\frac{x_{i}-y_{i}}{max_{i}-min_{i}} & \mbox{if numeric, else}\\ 0 &
\mbox{if $x_{i} = y_{i}$}\\ 1 & \mbox{if $x_{i} \neq y_{i}$}\\
\end{array} \right.
\end{math}

\item[-m 1] {\em Modified value difference metric} (Stanfill and
  Waltz, 1986; Cost and Salzberg, 1993).  This metric, usually
  referred to as {\sc mvdm}, estimates a numeric difference between
  two feature values from the difference of their cooccurrences with
  the classes of the training data. The distance between two values
  $V_{1},\ V_{2}$ of a feature is computed by taking the difference of
  the conditional distributions of the classes $C_{i}$ for these
  values: $\delta(V_{1}, V_{2}) = \sum_{i=1}^{n} \left| P(C_{i}|V_{1})
  - P(C_{i}|V_{2}) \right|$. With many values, keeping all {\sc mvdm}
  matrices in memory becomes computationally disadvantageous. {\sc
    fambl} determines by itself whether it will prestore {\sc mvdm}
  matrices, or whether it will compute {\sc mvdm} values on the fly.

\item[-m 2] {\em Jeffrey Divergence} is a statistical dissimilarity
  metric that can be used to compute the distance between class
  distributions of two values of the same feature. Functionally it is
  quite similar to {\sc mvdm}. It is best known for its application as
  a distance function in unsupervised vector space models, e.g. in
  image retrieval, where it is applied to histogram vectors. While
  {\sc mvdm} computes a straightforward geometrical distance between
  two class distribution vectors, Jeffrey divergence introduces a
  logarithm term:
\begin{math}
\delta(v_{1}, v_{2}) = \sum_{i=1}^{n} 
( P(C_{i}|v_{1}) log \frac{P(C_{i}|v_{1})}{m} +
  P(C_{i}|v_{2}) log \frac{P(C_{i}|v_{2})}{m} )
\label{jd}
\end{math}
Jeffrey divergence is a symmetric variant of Kullback--Leibler
distance; the $m$ term is used for this purpose:
\begin{math}
m = \frac{P(C_{i}|v_{1}) + P(C{i}|v_{2})}{2}
\label{jdm}
\end{math}
\end{description}
\ \\

\begin{tabular}{|p{0.1\textwidth}|p{0.3\textwidth}|p{0.52\textwidth}|}
\hline
{\tt -a} & & regard values of multi-valued features as atomic \\
\hline
\end{tabular}

The {\bf -a} switch ``unpacks'' multi-valued features into arrays of
atomic, or binary, features, where each binary feature denotes the
presence (value $1$) or absence (value $0$) of a particular value in
an instance. The weighting metric set by the {\bf -g} switch also
calculates weights for each of these individual
features. Hypothetically, assigning weights to atomic features can be
better than assigning weights to groups of values in the default
situation, since some feature values may be more important than
others. On the other hand, the number of observations on which a
weight is based can become dramatically smaller, and the weight can be
accordingly bad when used in classifying new material. See Van den
Bosch and Zavrel (2000) for a survey of the effects of unpacking
feature values.  \ \\

\begin{tabular}{|p{0.1\textwidth}|p{0.3\textwidth}|p{0.52\textwidth}|}
\hline
{\tt -C} & [1,32] (default 1) & set cardinality threshold of atomic feature combination search  \\
\hline
\end{tabular}

When the {\bf a} switch is selected, it is assumed by default that
atomic feature values should be discerned and weighted
individually. Setting {\bf -C} to higher values will make {\sc fambl}
search for {\em combinations} of atomic feature values, of cardinality
{\bf C}, that have a weight (set by the {\bf -g} switch) that is {\em
higher than the sum of the weights of the composing
parts}. Combinations are only made between atomic features from
different original multi-valued features. 

At cardinality 2, all possible combinations are tested. At higher
cardinalities $c$, only those compound features are tested that include
a strong compound at the previous cardinality $c-1$.

This option works very differently with different weighting metrics
(set with {\bf -g}). Good results (no dramatic explosion of numbers of
compund features, retaining generalisation accuracy) are generally
obtained with shared variance and chi square weighting.

Needless to say, this option slows down {\sc fambl} both in family
extraction and classification. Even more needless: setting {\bf -C}
has effect only when {\bf -a} is selected.
\ \\

\begin{tabular}{|p{0.1\textwidth}|p{0.3\textwidth}|p{0.52\textwidth}|}
\hline
{\tt -w} & [0,2] (default 0) & set family weighting metric \\
\hline
\end{tabular}

Families can be weighted individually with a weighting factor
$e$, an extra factor in the function that computes the distance
between a family $X$ and an instance to be classified $Y$:
$\Delta(X,Y) = e_{X} \sum_{i=1}^{n}\ w_{i}\
\delta(x_{i},y_{i})$. This makes sense when it is clear from the
learning material alone that certain families are bad predictors of
their own class, or very good ones. The bad ones should be suppressed
by low $e$ in the distance function; the good ones should be
boosted. {\sc fambl} allows the following variations on $e$:

\begin{description}

\item[-w 0 (default value)] All families have the same weight: for
each family $i$, $e_{i}=1.0$.

\item[-w 1] {\em Class prediction strength weighting} (Salzberg,
1991). The class prediction strength of a family $i$ is the number of
times the family is a nearest neighbor of a training instance
regardless of its class ($N$), minus the number of these nearest
neighbours that are of a different class ($Ndif_{i}$), divided by $N$
to express a portion between 0.0 and 1.0: $e_{i} =
\frac{N_{i}-Ndif_{i}}{N_{i}}$. A family with class-prediction
strength $e=1.0$ is a perfect predictor of its own class; an $e$ near
or at $0.0$ indicates that the family is a bad predictor.

\item[-w 2] {\em Class prediction strength weighting with Laplace
correction} (Nibblet, 1987; Clark and Nibblet, 1989; Domingos,
1996). Raw class prediction strength has a bias towards low-frequent
families and instances that is sometimes unwanted. The Laplace
correction introduces the number of classes, $c$, into the equation:
$e_{i} = \frac{(N_{i}-Ndif_{i})+1}{N_{i}+c}$. All other things being
equal, a larger absolute $N_{i}-Ndif_{i}$ leads to higher $e$.
\end{description}
\ \\

\begin{tabular}{|p{0.1\textwidth}|p{0.3\textwidth}|p{0.52\textwidth}|}
\hline
{\tt -b} & & perform basic {\sc mbl} ({\sc ib1}), i.e., create only 
single-instance families \\
\hline
\end{tabular}

{\bf -b} offers a sort of ``backward'' compatibility with pure
memory-based learning. It shuts down the family extraction stage, and
simply takes each individual instance (or instance type, if identical
instance tokens occur) to be its own family. It should be noted that
this option is generally slower than the implementation of
memory-based learning in TiMBL. There are small differences in Fambl's
implementation and TiMBL's, particularly in resolving class vote ties,
so that performances of {\sc fambl} with {\bf -b} might well deviate
from TiMBL's on the same data with the same algorithmic parameter
settings. \ \\

\begin{tabular}{|p{0.1\textwidth}|p{0.3\textwidth}|p{0.52\textwidth}|}
\hline
{\tt -k} & [1$>$ (default 1) & $k$ in $k$-NN classification \\
\hline
\end{tabular}

Although families in {\sc fambl} represent a kind of locally optimised
``precompiled'' $k$, which implies that $k$ in $k$-NN classification
is best set to 1, the {\bf -k} option offers the possibility to allow
matching to be based on more than one family (or rather more than one
``distance bin'', as each $k$ is seen as a bin in which all families
at a similar distance are grouped). 
\ \\

\begin{tabular}{|p{0.1\textwidth}|p{0.3\textwidth}|p{0.52\textwidth}|}
\hline
{\tt -K} & [1$>$ (default 1000) & $k$ in family extraction \\
\hline
\end{tabular}

By default, {\sc fambl} extracts families without any bounds. That
is. after having selected a starting instance for a family randomly,
{\sc fambl} builds the family expression by searching for all
same-class nearest neighbors until it finds a different-class
neighbor, or no more instances are available. Setting {\tt -K} to
lower values implies that the family extraction procedure generates
more instance-specific families; abstraction becomes more
``careful''. Setting {\tt -K} to 0 actually means emulating normal
{\sc ib1}, without any family extraction.  \ \\

\begin{tabular}{|p{0.1\textwidth}|p{0.3\textwidth}|p{0.52\textwidth}|}
\hline
{\tt -t} & {\sl typestring} & sets feature types according to typestring \\
\hline
\end{tabular}

Features can be either symbolic or numeric. Their type has
considerable effects on the metrics than can and cannot be applied to
them (e.g., {\sc mvdm} is designed for symbolic features). {\sc fambl}
selects the appropriate metrics, but needs to know first if a feature
is symbolic or numeric. This is encoded in the ``typestring''
string. This string must be a sequence of ``s''s and ``n''s; each
$n$th ``s'' (symbolic) or ``n'' (numeric) denotes the type of the
$n$th feature. For example, setting {\bf -t nnnss} declares the first
three features to be numeric, and the final two features to be
symbolic. 
\ \\

\begin{tabular}{|p{0.1\textwidth}|p{0.3\textwidth}|p{0.52\textwidth}|}
\hline
{\tt -n} & & declare all feature values as numeric \\
\hline
\end{tabular}

{\bf -n} is the shortcut for declaring all features to have numeric
values. If neither {\bf -t} or {\bf -n} is set, {\sc fambl} assumes
all features are symbolic. 
\ \\

\begin{tabular}{|p{0.1\textwidth}|p{0.3\textwidth}|p{0.52\textwidth}|}
\hline
{\tt -j} & & use joker (wildcard) values instead of value disjunctions 
(symbolic features) \\
\hline
\end{tabular}

Merging instances in {\sc fambl} means, by default, that mismatching
symbolic values are joined in a disjunction in the resulting family
expression. The alternative {\bf -j} introduces a ``joker value'', a
wildcard, that replaces any disjunction of two or more values of a
merged feature (as in {\sc rise}; Domingos, 1996). In classification,
the wildcard matches with any value. Usually this option leads to high
memory compression rates, not seldom at the cost of generalization
accuracy losses, because often individual values shouldn't be
``forgotten'', apparently. 
\ \\

\begin{tabular}{|p{0.1\textwidth}|p{0.3\textwidth}|p{0.52\textwidth}|}
\hline
{\tt -x} & [2>& collapse disjuncts to wildcards above $n$ values \\
\hline
\end{tabular}

Very long disjuncts at one feature in a family expression may indicate
that at that position it is actually not important what value occurs;
rather than an exhaustive list of values, a wildcard (joker) value may
be more appropriate for generalization. As a less absolute method than
the {\bf -j} option, with the {\bf -x} option it is possible to preset
a threshold of a number of values above which the disjunction is
replaced by a wildcard.  \ \\

\begin{tabular}{|p{0.1\textwidth}|p{0.3\textwidth}|p{0.52\textwidth}|}
\hline
{\tt -c} & & collapse hyperrectangles to into centroids (numeric features) \\
\hline
\end{tabular}

Hyperrectangles, the geometrical term for families of which the
features are numeric, are parts of feature space within which the
distance to instances falling in them is zero (Salzberg, 1991). With
{\bf -c}, hyperrectangles are collapsed into their geometrical mean,
the centroid, so that instances that are within the original
hyperrectangle space but not at the centroid are regarded as having a
non-zero distance to the family. This option usually compresses
memory. 
\ \\

\begin{tabular}{|p{0.1\textwidth}|p{0.3\textwidth}|p{0.52\textwidth}|}
\hline
{\tt -D} & & print NN distances along with classification output \\
\hline
\end{tabular}

At the end of each line in the automatically-generated output file
{\sl filestem.test$\ldots$out}, containing a test instance, its actual
class, and its predicted class, the distance of the nearest neighbour
is printed. If this distance is $0.0$, at least one of the nearest
neighours was an exact match of the test instance.
\ \\

\begin{tabular}{|p{0.1\textwidth}|p{0.3\textwidth}|p{0.52\textwidth}|}
\hline
{\tt -v} & [0,4] (default 1) & set verbosity level \\
\hline
\end{tabular}

{\bf -v} sets the level of verbosity of {\sc fambl}'s output. At {\bf
  -v0}, only a welcome message and the resulting generalisation score
is sent to {\tt stdout}. At {\bf -v2} and {\bf -v3}, incrementally
more data is printed during both learning and classification, but also
on the identity of the families being created. At {\bf -v4} {\sc
  fambl} is producing kilobytes of unintelligible debug-oriented
information.  \ \\

\begin{tabular}{|p{0.1\textwidth}|p{0.3\textwidth}|p{0.52\textwidth}|}
\hline
{\tt -o} & & write binary Fambl file filestem.fambl \\
\hline
\end{tabular}

For backup purposes, {\bf -o} writes the file {\sl filename}.fambl to
disk, where {\sl filename} is the name of the training file,
immediately after the family extraction stage. It can be read back in
a next session with {\bf -e}. The file is in a binary format.  \ \\

\begin{tabular}{|p{0.1\textwidth}|p{0.3\textwidth}|p{0.52\textwidth}|}
\hline
{\tt -e} & & read binary file filestem.fambl \\
\hline
\end{tabular}

{\bf -e} reads the file {\sl filename}.fambl back into memory,
skipping the family extraction stage. {\sl Filename} is the name of
the original training file, which does not need to be present when
reading the {\sc fambl} file.  \ \\

\begin{tabular}{|p{0.1\textwidth}|p{0.3\textwidth}|p{0.52\textwidth}|}
\hline
{\tt -r} & [1$>$ (default 1000) & rate at which to report on progress on 
processing families or test instances \\
\hline
\end{tabular}

{\bf -r} sets the rate at which {\sc fambl} reports on how many
families are extracted, or how many test instances are processed:
around every manyfold of the number set after {\bf -r}. Progress
reports on reading data files and computing class-prediction-strength
values during family extraction is done at a rate 100 times this
number.  \ \\

\section{Data format}
\label{format}

The training and test sets fed into {\sc fambl} consist of
descriptions of instances in terms of a fixed number of
feature-values. The files should contain one instance per line. The
number of instances, features, feature values and classes in the data
are determined automatically. The last feature of the instance is
assumed to be the target category. During testing, {\sc fambl} writes
the classifications of the test set to an output file. The format of
this output file is the same as the input format, with the addition of
the predicted category being appended after the correct category.

The data format demanded by {\sc fambl} is a derivative of the format
that is used by the well-known C4.5 decision tree learning program
(Quinlan, 1993).  The separator between the feature values can be
commas or white space (spaces or tabs). The class label (viz.  the
last feature on the line) may be followed by a period, but this is not
mandatory. Note that {\sc fambl} does not require a so called {\em
  namesfile}.

Example data files that come with the {\sc fambl} distribution, in the
{\tt demo} directory, are {\bf example-grapho.data} and {\bf
  example-grapho.test}. The top of {\bf example-grapho.data},
displayed below, exemplifies the comma-separated one-instance-per-line
{\sc c4.5} format.

\begin{quote}
{\tiny
\begin{verbatim}
_,_,_,_,a,b,e,r,r,(.
_,_,_,a,b,e,r,r,a,b.
_,_,a,b,e,r,r,a,t,^.
_,a,b,e,r,r,a,t,i,r.
a,b,e,r,r,a,t,i,o,-.
b,e,r,r,a,t,i,o,n,1.
e,r,r,a,t,i,o,n,_,-.
r,r,a,t,i,o,n,_,_,S.
r,a,t,i,o,n,_,_,_,-.
a,t,i,o,n,_,_,_,_,H.
_,_,_,_,a,b,o,v,e,^.
_,_,_,a,b,o,v,e,m,b.
_,_,a,b,o,v,e,m,e,V.
_,a,b,o,v,e,m,e,n,v.
a,b,o,v,e,m,e,n,t,-.
\end{verbatim}
$\ldots$
}
\end{quote}

The examples displayed here represent parts of words, in the form of
fixed windows of nine letters. The tenth field, i.e., the final
comma-separated column, represents the pronunciation (in phonemic
code) of the middle letter. More generally, the last field always
represents the class; all other columns represent features.

\section{An annotated example run}
\label{examples}

Using the example data files {\bf example-grapho.data} and {\bf
  example-grapho.test} (included in the package's {\tt demo}
directory), {\sc fambl} is run with the following commandline:

\begin{quote}
{\footnotesize
\begin{verbatim}
> Fambl -f example-grapho.data -t example-grapho.test
\end{verbatim}
}
\end{quote}

This means that {\sc fambl} will learn from {\bf example-grapho.data},
and test on {\bf example-grapho.test}.  The full log resulting from
this command is the following:

\begin{quote}
{\tiny
\begin{verbatim}
Fambl (Family-based learning), version 2.2, 26 July 2004
(c) 1997-2004 ILK Research Group, Tilburg University
http://ilk.uvt.nl / antalb@uvt.nl
current time: Mon Jul 26 22:59:25 2004
     metric scheme set to GR feature wgts, no distance wgts, no MVDM
     sorting and reading data base example-grapho.data
     data base has 6124 instances
                   5971 instance types
                   9 features
                   57 classes
     computing feature gain ratio values
     feature   0 (   28 values):   0.078196
     feature   1 (   28 values):   0.085658
     feature   2 (   30 values):   0.100040
     feature   3 (   31 values):   0.190212
     feature   4 (   30 values):   0.720676
     feature   5 (   31 values):   0.224557
     feature   6 (   30 values):   0.127532
     feature   7 (   29 values):   0.095540
     feature   8 (   29 values):   0.087888
     average gain ratio: 0.190033
     presorting and rereading instance base example-grapho.data
     took 3 seconds
 <*> family extraction stage started
         1000 families (covering 80.81%),    4.95 members av.
     # families                        :      1677
     average # members                 :    3.6518
     average description length (bytes):  113.5504
     compression (raw memory)          :   22.2632 %
     compression (vs instance types)   :   27.5194 %
     #type vs. #fam reduction          :   71.9143 %
     clusteredness                     :    102.03
     took 2 seconds
 <*> starting test with k=1
     writing output to example-grapho.test.Fambl.O.gr.k1.K1000.out
         501 instances out of     657 classified correctly
     Fambl score:  76.2557 % correct instances
     took 0 seconds
     (all 657 instances within one second)
 <*> current time: Mon Jul 26 22:59:30 2004
     Fambl spent a total of 5 seconds running;
     5 on learning, 0 on testing.
     Fambl ready.
\end{verbatim}
}
\end{quote}

This log should be read as follows:

\begin{quote}
{\tiny
\begin{verbatim}
Fambl (Family-based learning), version 2.2, 26 July 2004
(c) 1997-2004 ILK Research Group, Tilburg University
http://ilk.uvt.nl / antalb@uvt.nl
current time: Mon Jul 26 22:59:25 2004
     metric scheme set to GR feature wgts, no distance wgts, no MVDM
     sorting and reading data base example-grapho.data
     data base has 6124 instances
                   5971 instance types
                   9 features
                   57 classes
     computing feature gain ratio values
     feature   0 (   28 values):   0.078196
     feature   1 (   28 values):   0.085658
     feature   2 (   30 values):   0.100040
     feature   3 (   31 values):   0.190212
     feature   4 (   30 values):   0.720676
     feature   5 (   31 values):   0.224557
     feature   6 (   30 values):   0.127532
     feature   7 (   29 values):   0.095540
     feature   8 (   29 values):   0.087888
     average gain ratio: 0.190033
     presorting and rereading instance base example-grapho.data
     took 3 seconds
\end{verbatim}
}
\end{quote}

{\sc Fambl} starts with an echo of its name and version, and a time
stamp. It reports on the selected metrics (either by default or by
user override). Since the command line did not specify any metrics,
{\sc fambl} reports it will use information gain ratio (GR) feature
weighting and the overlap metric. It then proceeds to read the training
file {\bf example-grapho.data}, finding out how many instances (both
instance tokens and types), features, feature values, and classes are
in this file. Having established that, {\sc fambl} computes the
information gain ratio feature weights of the nine features. The final
initialisation step is to re--sort and re--read the data into
memory. This completes all initialisation, taking 3 seconds. {\sc
fambl} then starts the family extraction stage:

\begin{quote}
{\tiny
\begin{verbatim}
 <*> family extraction stage started
         1000 families (covering 80.81%),    4.95 members av.
     # families                        :      1677
     average # members                 :    3.6518
     average description length (bytes):  113.5504
     compression (raw memory)          :   22.2632 %
     compression (vs instance types)   :   27.5194 %
     #type vs. #fam reduction          :   71.9143 %
     clusteredness                     :    102.03
     took 2 seconds
\end{verbatim}
}
\end{quote}

{\sc Fambl} reports at each manifold of 1000 extracted families
how much of the original instance base is being covered already by
extracted families, and how many members on average are in the
extracted families. Since the largest families are extracted first
generally, this average tends to drop during extraction. Upon
completion, {\sc fambl} reports on the number of families (in this
example, 1,677); the average number of family members (over 3.65),
the average length in bytes needed to store the families in memory
(114), and three compression rates:

\begin{enumerate}

\item
{\em compression (raw memory)} is the compression rate in memory
needed for storing all families, as compared to storing all instances
in pure memory-based learning. This is the most concrete compression
rate, which is not seldom negative because of the fact that storing
families involves exceedingly more bookkeeping (e.g., storing
disjunctions rather than singular values) than storing instances for
which the number of features and values is fixed and known.

\item
{\em compression (vs. instance types)} compares {\sc fambl}'s storage
requirements against the requirements of storing instance types with
frequency counts.

\item
{\em \#type vs. \#fam reduction} is the percentage reduction of the
number of families versus the number of instance types. This is never
negative, and provides generally the most flattering compression
figures, abstracting of course over the actual numbers of bytes it
takes to store families versus instance types.
\end{enumerate}

{\sc fambl} then reports on {\em clusteredness}, which is the weighted
average of the number of clusters in which classes are scattered. The
higher this number, the more disjunct a data set is. This concludes
the family extraction stage, which took 2 seconds. The subsequent test
stage is then started:

\begin{quote}
{\tiny
\begin{verbatim}
 <*> starting test with k=1
     writing output to example-grapho.test.Fambl.O.gr.k1.K1000.out
         501 instances out of     657 classified correctly
     Fambl score:  76.2557 % correct instances
     took 0 seconds
     (all 657 instances within one second)
\end{verbatim}
}
\end{quote}

The example test file {\bf example-grapho.test} contains 657
instances, of which 501 are classified correctly, yielding an accuracy
percentage of 76.3\%. Testing took less than one second. All output of
{\sc fambl} is written during testing to a file called {\bf
  example-grapho.test.Fambl.O.gr.k1.K1000.out} ({\bf O}: weighted
overlap, {\bf gr}: information gain ratio, {\bf k1}: $k=1$, {\bf
  K1000}: $K=1000$). The top of this file looks as follows:

\begin{quote}
{\tiny
\begin{verbatim}
_,_,_,_,e,c,l,a,i,1,r
_,_,_,e,c,l,a,i,r,k,k
_,_,e,c,l,a,i,r,_,l,l
_,e,c,l,a,i,r,_,_,8,1
e,c,l,a,i,r,_,_,_,-,-
c,l,a,i,r,_,_,_,_,R,R
_,_,_,_,a,e,r,o,p,8,-
_,_,_,a,e,r,o,p,l,-,-
_,_,a,e,r,o,p,l,a,r,r
_,a,e,r,o,p,l,a,n,^,^
a,e,r,o,p,l,a,n,e,p,p
e,r,o,p,l,a,n,e,s,l,l
r,o,p,l,a,n,e,s,_,1,-
o,p,l,a,n,e,s,_,_,n,n
p,l,a,n,e,s,_,_,_,-,-
\end{verbatim}
$\ldots$
}
\end{quote}

This file provides all test instances, with an additional eleventh
column containing the class predicted by {\sc fambl}.

\begin{quote}
{\tiny
\begin{verbatim}
 <*> current time: Mon Jul 26 22:59:30 2004
     Fambl spent a total of 5 seconds running;
     5 on learning, 0 on testing.
     Fambl ready.
\end{verbatim}
}
\end{quote}

When testing is finished, {\sc fambl} summarises how much time was
spent on learning (reading data plus extracting families) and testing,
and exits.

\section{Limitations}
\label{notes}

In {\sc fambl}'s implementation certain choices were made that limit
certain aspects of the data that can be handled by {\sc fambl}. All
restrictions are set in the {\tt \#define}s in the file {\tt
Common.h}, and may be changed before compilation by the user at own
risk. In short, the restrictions are the following:

\begin{itemize}
\item
the number of features may not exceed 2048.
\item
the number of feature values may not exceed {\tt INT\_MAX}. This is
more than usually necessary. To save some memory, provided that all of
the features in your dataset have less than 32,768 values, change the
type definition of {\tt value} from {\tt int} to {\tt short int}.
\item
the number of classes is not allowed to exceed 8192.
\item
{\sc fambl} cannot extract more than 1,500,000 families.
\end{itemize}

Finally, the following list of problems is known to occur. In general,
please take care (especially of available disk space) when applying
{\sc fambl} to very big data files. Please send bug reports to  {\tt
Antal.vdnBosch@uvt.nl}.

\begin{itemize}
\item
{\sc Fambl} writes temporary files in the current working directory. When
{\sc fambl} is interrupted before family extraction, all temporary material
is left sitting there. Be sure to delete any {\tt fambl*tmp} files after
interrupted runs.
\item
Some number overflow checks may not be waterproof yet. Extreme data may
prove this. The biggest data set to date tested successfully contained
2.5 million cases with 11 symbolic features.
\end{itemize}

\subsection*{Acknowledgements}

Thanks go out to Walter Daelemans, Jakub Zavrel, Iris Hendrickx, Ton
Weijters, Eric Postma, Jorn Veenstra, Sabine Buchholz, Bertjan Busser,
Ko van der Sloot, Hans van Halteren, Gert Durieux, Stephan
Raaijmakers, Jaap van den Herik, David Aha, Pedro Domingos, Dietrich
Wettschereck, and Fred Wan. Thanks to the Tilburg ILK group for
comments, moral and technical support.

\section*{References}
\label{refs}

\begin{description}
\item
  Aha, D. W., D. Kibler, and M. Albert (1991). Instance-based 
    learning algorithms. {\em Machine Learning}, 6:37-66.
\item
  Clark, P. and T.~Niblett. (1989). The {CN}2 rule induction algorithm.
    {\em Machine Learning}, 3:261--284.
\item
  Cost, S. and S. Salzberg (1993). A weighted nearest neighbour 
    algorithm for learning with symbolic features.
    {\em Machine Learning}, 10:57-78.
\item
  Cover, T. M. and P. E. Hart (1967). Nearest neighbor pattern 
    classification. {\em Institute of Electrical and Electronics 
    Engineers Transactions on Information Theory}, 13:21-27.
\item
  Daelemans, W., A. van den Bosch, and A. Weijters (1997). IGTree: 
    using trees for compression and classification in lazy learning 
    algorithms. {\em Artificial Intelligence Review}, 11:407-423.
\item
  Daelemans, W., A. van den Bosch, and J. Zavrel (1997). A 
    feature-relevance heuristic for indexing and compressing large 
    case bases. In M. van Someren and G. Widmer, {\em Poster Papers of 
    the Ninth European Conference on Machine Learing}, University of 
    Economics, Prague, Czech Republic, pages 29-38.
\item
  Daelemans, W., A. van den Bosch, and J. Zavrel (1999).
    Forgetting exceptions is harmful in language learning.
    {\em Machine Learning}, 11:11-43.
\item
  Domingos, P. (1996). Unifying instance-based and rule-based induction.
    {\em Machine Learning}, 24:141--168.
\item
  Dunning, T. (1993). Accurate methods for the statistics of surprise 
    and coincidence. {\em Computational Linguistics}, 19:1, 61--74.
\item
  Niblett, T. (1987). Constructing decision trees in noisy domains.
    In {\em Proceedings of the Second European Working Session on
    Learning}, Bled, Yugoslavia, pp. 67--78.
\item
  Quinlan, J. R. (1993). {\em C4.5: Programs for Machine Learning}.
    Morgan Kaufmann, San Mateo, CA.
\item
  Salzberg, S. (1991). A nearest hyperrectangle learning method.
    {\em Machine Learning}, 6:277--309.
\item
  Stanfill, C. and D. Waltz (1986). Toward memory-based reasoning.
    {\em Communications of the ACM}, 29(12):1213--1228.
\item
  Van den Bosch, A (1997). {\em Learning to pronounce written words.
    A study in inductive language learning}. Ph.D. thesis, 
    Universiteit Maastricht, Maastricht, The Netherlands.
\item
  Van den Bosch, A. (1999a). Instance-family abstraction in memory-based
    language learning. In I. Bratko and S. Dzeroski (Eds.), {\em Machine
    Learning: Proceedings of the Sixteenth International Conference},
    ICML'99, Bled, Slovenia, pp. 39--48.
\item
  Van den Bosch, A. (1999b). Careful abstraction from instance
    families in memory-based learning. To appear in W. Daelemans
    (guest ed.), {\em Journal of Experimental and Theoretical
    Artificial Intelligence}, special issue on memory-based learning
    of language processing.
\item
  Van den Bosch, A. and Zavrel, J. (2000). Unpacking multi-valued 
    symbolic features and classes in memory-based language learning.
    In P. Langley (Ed.), {\em Proceedings of the Seventeenth International
    Conference on Machine Learning}, pp. 1055-1062. San Francisco, 
    CA: Morgan Kaufmann, 2000.
\end{description}

\end{document}
