% MBT 1.0 manual
%
% main text by Walter spring-summer 2002
% some edits Antal Oct 2002

\documentclass{report}

\usepackage{epsf}
\usepackage{a4wide}
\usepackage{palatino}
\usepackage{fullname}

\newcommand{\chisq}{{$ \chi^2 $}}

\author{Walter Daelemans* \and Jakub Zavrel*$\dagger$ \and
	Antal van den Bosch \and Ko van der Sloot\\ \ \\
	Induction of Linguistic Knowledge\\
	Computational Linguistics\\ 
        Tilburg University \\ \\
	(*) CNTS - Language Technology Group\\
	University of Antwerp\\ \\
	($\dagger$) Textkernel B.V.\\ \\
        P.O. Box 90153, NL-5000 LE, Tilburg, The Netherlands \\ 
        URL: http://ilk.uvt.nl\thanks{This document is available from
	http://ilk.uvt.nl/downloads/pub/papers/ilk.0209.ps. All rights reserved
	Induction of Linguistic Knowledge, Tilburg University and 
        CNTS Research Group, University of Antwerp.}}

\title{{\huge MBT: Memory-Based Tagger} \\ \vspace*{0.5cm}
{\bf version 1.0} \\ \vspace*{0.5cm}{\huge Reference Guide}\\
\vspace*{1cm} {\it ILK Technical Report -- ILK 02-09}}

%better paragraph indentation
\parindent 0pt
\parskip 9pt

\begin{document}

\pagenumbering{roman} 

\maketitle

\tableofcontents

\chapter*{Preface}

Part-of-Speech (POS) tagging is a process in which a morpho-syntactic
class is assigned to each word in a text on the basis of the
characteristics of the word and of the context in which it occurs. It
is a first level of abstraction in text analysis, and is used in many
language technology applications such as (shallow) parsing,
information retrieval, spelling error correction, speech synthesis, and
text mining. 

Memory-Based Tagging is an approach to POS Tagging based on {\em
Memory-Based Learning} ({\sc mbl}).  As an adaptation and extension of
the classical $k$-Nearest Neighbor ($k$-NN) approach to statistical
pattern classification, {\sc mbl} has proven to be successful in a
large number of tasks in Natural Language Processing (NLP). Since
1998, we have made available TiMBL, a flexible software tool
incorporating an extensive family of memory-based learning
algorithms. The most recent version of software and documentation are
available free for research and teaching from {\tt http://ilk.uvt.nl}.
To use this software for POS Tagging is not straightforward,
however. We want to be able to use previous tagger decisions as input
for current decisions, we want to build separate case bases for known
and unknown words, allow global sentence-level optimization, etc.  The
software you will find here implements this specific tagging
functionality by wrapping software around TiMBL while keeping
most of the flexibility of TiMBL intact. 

Memory-Based Tagging was originally proposed in
\cite{Daelemans95}. The most complete description to date is contained
in the union of \cite{Daelemans+96} and \cite{Zavrel+99}.  As is the
case for TiMBL, the main effort in the development and
maintenance of this software was invested by Ko van der Sloot. The
system started as a rewrite of code developed by Peter Berck and
adapted by Jakub Zavrel. The code has benefited substantially from
trial, error and scrutiny by past and present members of the {\sc ilk}
and {\sc cnts} groups in Tilburg and Antwerp. This software was
written in the context of projects funded by the Netherlands
Organization for Scientific Research (NWO), Tilburg University Faculty
of Arts, the Flemish National Science Foundation (FWO), and the
University of Antwerp Research Council.

The current release (version 1.0) is the first public release, and
uses TiMBL version 4.3. Note that you have to download and
license your personal download of TiMBL 4.3 separately, if you
have not already done so, to be able to compile the tagging
software. See {\tt http://ilk.uvt.nl}.

This reference guide is structured as follows. In
Chapter~\ref{license} you can find the terms of the license according
to which you are allowed to use {\sc mbt}. The subsequent chapter
gives instructions on how to install the software on your computer.
Next, Chapter~\ref{tutorial} offers a tutorial and information about
the different parameters of the system.  Readers who are interested in
the theoretical and technical details of Memory-Based Tagging should
consult \cite{Daelemans+96,Zavrel+99}. These papers are also included
in this software distribution.

This document does {\em not} contain information about the TiMBL
software package. In order to make the best use of this tagging software,
it is strongly advised to get acquainted with the functionality of
TiMBL, as explained in its reference guide \cite{Daelemans+02}.

\chapter{License terms}
\label{license}
\pagenumbering{arabic} 

Downloading and using the MBT software implies that you accept the
following license terms:

Tilburg University and University of Antwerp (henceforth
``Licensers'') grant you, the registered user (henceforth ``User'')
the non-exclusive license to download a single copy of the MBT
program code and related documentation (henceforth jointly referred to
as ``Software'') and to use the copy of the code and documentation
solely in accordance with the following terms and conditions:

\begin{itemize}

\item The license is only valid when you register as a user. If you
have obtained a copy without registration, you must immediately
register by sending an e-mail to {\tt Mbt@uvt.nl}.

\item User may only use the Software for educational or non-commercial
research purposes.

\item Users may make and use copies of the Software internally for
their own use.

\item Without executing an applicable commercial license with
Licensers, no part of the code may be sold, offered for sale, or made
accessible on a computer network external to your own or your
organization's in any format; nor may commercial services utilizing
the code be sold or offered for sale. No other licenses are granted or
implied.

\item Licensers have no obligation to support the Software it is
providing under this license.  To the extent permitted under the
applicable law, Licensers are licensing the Software "AS IS", with no
express or implied warranties of any kind, including, but not limited
to, any implied warranties of merchantability or fitness for any
particular purpose or warranties against infringement of any
proprietary rights of a third party and will not be liable to you for
any consequential, incidental, or special damages or for any claim by
any third party.

\item Under this license, the copyright for the Software remains the
joint property of the ILK Research Group at Tilburg University, and
the CNTS Research Group at the University of Antwerp.  Except as
specifically authorized by the above licensing agreement, User may not
use, copy or transfer this code, in any form, in whole or in part.

\item Licensers may at any time assign or transfer all or part of their
interests in any rights to the Software, and to this license, to an
affiliated or unaffiliated company or person.

\item Licensers shall have the right to terminate this license at any
time by written notice. User shall be liable for any infringement or
damages resulting from User's failure to abide by the terms of this
License.

\item In publication of research that makes use of the Software, a
citation should be given of: {\em ``Walter Daelemans, Jakub Zavrel,
Antal van den Bosch and Ko van der Sloot (2002). MBT: Memory-Based
Tagger, Reference Guide. ILK Technical Report 02-09, \\
Available from
{\tt http://ilk.uvt.nl/downloads/pub/papers/ilk.02-09.ps}}

\item For information about commercial licenses for the Software,
contact {\tt Mbt@uvt.nl}, or send your request to:

Prof. dr.~Walter Daelemans\\
CNTS - Language Technology Group\\
GER / University of Antwerp\\
Universiteitsplein 1\\
B-2610 Wilrijk (Antwerp)\\
Belgium\\
Email: daelem@uia.ua.ac.be

\end{itemize}

\pagestyle{headings}

\chapter{Installation}
\vspace{-1cm}
You can get the {\sc mbt} package as a gzipped tar archive from:

{\tt http://ilk.uvt.nl/software.html}

Following the links from that page, you will be required to fill in
registration information and to accept the license agreement. You can
proceed to download the file {\tt Mbt.1.0.tar.gz}. This file contains
the complete source code (C++) for the {\sc mbt} program, a sample
data set, the license and the documentation. The installation should
be relatively straightforward on most UNIX systems.

To install the package on your computer, unzip the downloaded file:

{\tt > gunzip Mbt.1.0.tar.gz}

and unpack the tar archive:

{\tt > tar -xvf Mbt.1.0.tar}

This will make a directory {\tt Mbt.1.0} under your 
current directory. Change directory to this:

{\tt > cd Mbt.1.0}

and edit the variable {\tt TIMBLPATH} in the {\tt Makefile}. This
variable should be set to the directory where a {\em compiled} Timbl
4.3 system resides.

Compile the executable by typing {\tt make}\footnote{We have
tested this with {\tt gcc} versions 2.95.2, 2.95.3, and 3.0.3},
assuming that {\tt make} is actually {\tt gnumake} on your system.
Solaris users should use {\tt gnumake} explicitly, since their {\tt
make} defaults to SunOS {\tt make}, which does not operate correctly.

If the process was completed successfully, you should now have
executable files named {\tt Mbtg, Mbt}, and a static library {\tt
libMbt.a}.

The e-mail address for problems with the installation, bug reports,
comments and questions is {\tt Mbt@uvt.nl}.

\chapter{Tutorial and Reference}
\label{tutorial}

Memory-based tagging is based on the idea that words occurring in
similar contexts will have the same POS tag. The idea is implemented
using the memory-based learning software package TiMBL ({\tt
http://ilk.uvt.nl/software.html}).  The {\sc mbt} software package
makes use of TiMBL to implement a Part of Speech (POS)
tagger--generator.  The software consists of two executables: {\tt
Mbtg} to generate a tagger, and {\tt Mbt} to use a generated tagger on
text data.  Given as input an annotated (tagged) corpus, {\tt Mbtg}
generates a lexicon, and case bases for known words (words in the
corpus, hence also in the lexicon), and unknown words (in order to
guess the POS tag of words not in the corpus). The lexicon associates
words with their ambiguous tag, henceforth referred to as {\em
ambitag}: a symbol representing all the POS tags a word can have
according to the corpus. The {\tt Mbt} executable takes a tagger
constructed by {\tt Mbtg} as input and can be used to tag text with
it.  For theoretical background, see \cite{Daelemans+96,Zavrel+99}.

This document exemplifies how to use the MBT package. As an example
data file, we have added a small part of the tagged ``Eindhoven
Corpus'' of Dutch written text \cite{Uitdenboogaard75} to the
distribution, a Dutch POS-tagged corpus. The data consists of only
about 100,000 words, so the quality of taggers trained with this data
will not be high. It is meant as a toy corpus. The tag set used
consists only of 10 broad-category POS tags.

\section{{\tt Mbtg}, the tagger generator}

The input file containing the material for generating a tagger must
consist of two whitespace-separated columns. The first column contains
a word or punctuation mark with in the corresponding position of the
second column its POS tag. A line may also contain only the symbol
{\tt $<$utt$>$} to mark the end of a sentence. The following are the two first
sentences of the file {\tt eindh.data}, present in this distribution
of {\sc mbt}.

{\footnotesize
\begin{verbatim}
Dit	Pron
in	Prep
verband	N
met	Prep
de	Art
gemiddeld	Adj
langere	Adj
levensduur	N
van	Prep
de	Art
vrouw	N
.	Punc
<utt>
De	Art
verzekeringsmaatschappijen	N
verhelen	V
niet	Adv
dat	Conj
ook	Adv
de	Art
rentegrondslag	N
van	Prep
vier	Num
procent	N
nog	Adv
een	Art
ruime	Adj
marge	N
laat	V
ten	Prep
opzichte	N
van	Prep
de	Art
thans	Adv
geldende	V
rentestand	N
.	Punc
<utt>
\end{verbatim}
}


\subsection{Defining feature patterns}

In generating the tagger, information has to be provided to the
tagger generator about the context and the form of the words to be
tagged. This is done by the parameters {\tt -p} (feature pattern for known
words), and {\tt -P} (feature pattern for unknown words). Patterns are
built up as combinations of the following symbols:

\subsubsection*{For -p and -P}

\begin{tabular}{ll}
{\tt d} & Left context (tag)\\
{\tt a} & Right context (ambitag)\\
{\tt w} & Left or right context (word)\\
\end{tabular}

\subsubsection*{For -p only (known words)}

\begin{tabular}{ll}
{\tt f} & Focus (ambitag for known words)\\
{\tt W} & Focus (word)\\
\end{tabular}

\subsubsection{For -P only (unknown words)}

\begin{tabular}{ll}
{\tt F} & Focus (position of the unknown word) \\
{\tt c} & The focus contains capitalized characters \\
{\tt h} & The focus word contains a hyphen\\
{\tt n} & The focus word contains numerical characters \\
{\tt p} & Character at the start of the word \\
{\tt s} & Character at the end of the word \\
\end{tabular}

The symbols {\tt d}, {\tt a}, {\tt w}, {\tt p}, and {\tt s} can occur
more than once to indicate the scope of the context. Symbols to the
left of the focus symbols indicate left context, and symbols to the
right of the focus symbols indicate right context.

For example, for known words, the following are a few possible patterns: 

\begin{tabular}{ll}
{\tt dfa}  & focus ambitag with one disambiguated tag on the left and one ambitag to the right \\
{\tt ddfa} & focus ambitag with two disambiguated tags to the left and one ambitag to the right \\ 
{\tt ddfWa} & as previous, plus the focus word (note that {\tt W} can be declared only immediately after {\tt f}) \\
{\tt dwdwfWaw} & as previous, plus for each context tag the corresponding word (two left, one right)\\ 
\end{tabular}

For unknown words:

\begin{tabular}{ll}
{\tt dFa} & one disambiguated tag to the left and one ambitag to the right \\
{\tt psdFa} & as previous, plus the first and last letter of the unknown word to be tagged \\ 
{\tt psssdFa} & as previous, plus the three last letters of the word to be tagged \\
{\tt psssdwFaw} & as previous, plus the left and right neighboring words \\
\end{tabular}

An example command line for tagger generation is the following (``{\tt
$>$}'' is the command line prompt):

{\small
\begin{verbatim}
 > Mbtg -T eindh.data -p ddfa -P psssdFa
\end{verbatim}
}

This will generate a tagger based on information about the previous
two predicted tags and the following ambitag for the known words, and
about the first and three last letters of the word, the previous predicted
tag, and the following ambitag for the unknown words. Supposing the
annotated corpus you use to construct the tagger is in the two-column
file {\tt eindh.data}, Mbtg will generate the following output
to standard output:

{\small
\begin{verbatim}
Memory Based Tagger Generator Version 1.0
  (c) ILK and CNTS 1998 - 2002.
  Induction of Linguistic Knowledge Research Group, Tilburg University
  Centre for Dutch Language and Speech, University of Antwerp

  Based on Timbl version 4.3

Constructing a tagger from: eindh.data
  Creating lexicon: eindh.data.lex of 17044 entries.
\end{verbatim}
}

The first data structure created by the tagger is a frequency-sorted
lexicon {\tt eindh.data.lex} with for each word the different tags it
was assigned, along with their frequency in the training corpus. 

{\small
\begin{verbatim}
  Creating ambitag lexicon: eindh.data.lex.ambi.05
  Creating ambitag translation table: eindh.data.ambi.05
\end{verbatim}
}

An ambitag is a symbolic label defining for a word which different
tags it can have according to the corpus.  In the ambitag lexicon,
each word is associated with its corresponding ambitag, represented in
two forms: a letter code generated by the tagger, and a string of tags
separated by hyphens. Some frequency-based smoothing is implemented in
this approach: whenever a word--tag combination occurs less than a
given percentage (5\% by default) of the word's total frequency, it is
not included in the ambitag. The parameter {\tt -\%} $<percentage>$ can be
used to modify this threshold. The ambitag translation table is used
for associating the generated ambitag letter codes with the more
understandable hyphenated representation.

{\small
\begin{verbatim}
  Creating list of most frequent words: eindh.data.top100
\end{verbatim}
}

Next, the tagger generator creates a list with (by default) the 100
most frequent words in the corpus. Only words in this list will be
used when the symbols {\em w, W} are used in the {\tt -p}, {\tt -P}
patterns. The number of most frequent words can be modified with the
parameter {\tt -M} $<number>$. All words {\em not}\/ in the
most-frequent-words list are transformed into special symbols: {\tt
HAPAX-}$<code>$, where $<code>$ is either {\tt 0}, or a combination of
{\tt H}, {\tt C}, and {\tt N}. {\tt H} indicates that the word
contains a hyphen, {\tt C} denotes that the word is capitalised, and
{\tt N} indicates that the word contains a number. {\tt HAPAX-HCN}
would for example be the transformed code for the word ``B-52''.

{\small
\begin{verbatim}
  Create known words case base
    Timbl options: ' -a IGTREE -vS   -H'
    Algorithm = IGTREE

    Processing data from the file eindh.data..................................
      ready: 95548 words processed.
    Creating case base: eindh.data.known.ddfa
    Deleted intermediate file: eindh.data.known.inst.ddfa
\end{verbatim}
}

In this part of the tagger generation process, the case base for known
words is generated. To do this, TiMBL is used, by default with
the options shown above ({\sc igtree} algorithm for known
words). The process consists of two steps. First, instances are
created using the specified information sources for known words (as
indicated in {\tt -p}), then the case base is generated from that (which
may imply a significant storage reduction, depending on the {\sc
timbl} options used). Finally, the intermediate file with instances is
deleted --- this can be overruled with the option {\tt -X}.

{\small
\begin{verbatim}
  Create unknown words case base
    Timbl options: ' -a IB1 -vS  -H'
    Algorithm = IB1

    Processing data from the file eindh.data..................................
      ready: 95548 words processed.
    Creating case base: eindh.data.unknown.psssdFa
    Deleted intermediate file: eindh.data.unknown.inst.psssdFa
\end{verbatim}
}

This part of the screen log describes the process of the creation of
the case base for unknown words (for which no ambitag is available,
and for which the inclusion of the word itself in the input is
pointless). It is parallel to the procedure for known words, but it
uses information sources specified in the {\tt -P} pattern, and uses
as default {\sc timbl} settings the {\sc ib1-gr} algorithm (the
overlap metric with gain ratio feature weighting).

{\small
\begin{verbatim}
  Created settings file 'eindh.data.settings'

Ready:
  Time used: 83
  Words/sec: 2302
\end{verbatim}
}

The tagger generation process ends with some information about the
real time needed to construct the tagger (total time used and number
of words per second), and with the construction of a settings file,
which will be used by the Mbt executable to use the tagger on
new data. E.g. for our toy corpus, the settings file looks like this:

{\small
\begin{verbatim}
e <utt>
l eindh.data.lex.ambi.05
k eindh.data.known.ddfa
u eindh.data.unknown.psssdFa
r eindh.data.ambi.05
p ddfa
P psssdFa
O K: -a IGTREE -vS U: -a IB1 -vS
L eindh.data.top100
\end{verbatim}
}

These settings can be modified by the user to a certain extent, which
is especially useful for experimenting with different TiMBL
options. However, this is not advised unless the user knows what s/he
is doing (not all TiMBL options and Mbtg settings can be
modified given the datastructures created).

Some Mbtg parameters have remained undiscussed until now. 

\begin{itemize}
\item 
In the construction of the unknown words case base, instances are
created only for these words which are relatively infrequent (the
basic idea here is that unknown words will behave similarly to
infrequent known words).  The option {\tt -n} $<n>$ determines the maximum
frequency a word can have in the training corpus to use it (and its
context) in the creation of the unknown words case base (default = 5).
\item
By default, the tagger generator expects the symbol {\tt <utt>} on a line
to indicate the boundary between two sentences.
Using the option {\tt -e} $<string>$, this default behaviour can be
modified. With {\tt -e NL}, one or more empty lines between word-tag lines
would be interpreted as an utterance marker.
\end{itemize}

\section{{\tt Mbt}, the tagger}

Given that {\tt Mbtg} was used to generate data files and a settings file
defining a memory-based tagger, {\tt Mbt} can be used to tag text. For
example, continuing our example:

{\small
\begin{verbatim}
Mbt -s eindh.data.settings -t <testfile>
\end{verbatim}
}

This starts the memory-based tagger, and sends the tagged output to
standard output. The testfile should be tokenized (i.e. punctuation
marks should be separated from the words). E.g. we included a short
tokenized test file in the distribution (file {\tt test.tok}).

{\small
\begin{verbatim}
> Mbt -s eindh.data.settings -t test.tok > test.out

Memory Based Tagger Version 1.0 (beta)
  (c) ILK and CNTS 1998 - 2002.
  Induction of Linguistic Knowledge Research Group, Tilburg University
  Centre for Dutch Language and Speech, University of Antwerp

  Based on Timbl version 4.3
  Sentence delimiter set to '<utt>'
  Beam size = 1
  Known Tree, Algorithm = IGTREE
  Unknown Tree, Algorithm = IB1

  Reading the ambitags from: eindh.data.ambi.05...ready, (78 tags).
  Reading the lexicon from: eindh.data.lex.ambi.05...ready, (17045 words).
  Reading frequent words list from: eindh.data.top100...ready, (100 words).
  Reading case-base for known words from: eindh.data.known.ddfa... ready.
  Reading case-base for unknown words from: eindh.data.unknown.psssdFa... ready.

Processing data from the file test.tok. ready.

Done:
  32 words processed.
  Known   words: 20
  Unknown words: 12 (37.5 %)
  Total        : 32
  Time used: 2
  Words/sec: 16
\end{verbatim}
}

The file {\tt test.out} will now contain the tagged text:

{\small 
Het/Art Centrum//N voor/Prep Nederlandse/Adj Taal//Adv en/Conj
Spraak//N en/Conj de/Art Inductie//N van/Prep Linguistische//Adj
Kennis//N onderzoeksgroep//V werken/V aan/Adv geheugengebaseerd//V
leren/V voor/Adv taaltechnologie//N ./Punc Al/Adv in/Prep 1994//Num
werd/V er/Adv gewerkt/V aan/Prep de/Art memory-based//N tagger//N
./Punc
}

In producing output, the tagger concatenates tags to words,
separated by either a single slash (/) when the word is in the
lexicon, or a double slash (//) when the unknown words case base was
used to predict the tag.

With {\tt -T} $<testfile>$, files in the format of the tagger generation
input (one word per line, two columns for word and corresponding tag)
can be used. In that case, accuracy figures are computed, and the
output, again sent to standard output, consists of the two input
columns, separated by a / or //, and an additional column with the
predicted tag. This way, a constructed tagger can be evaluated. E.g., 

{\small
\begin{verbatim}
Mbt -s eindh.data.settings -T eindh.test > eindh.test.out
\end{verbatim}
}

will send the tagged text (four columns) to {\tt eindh.test.out},
and compute total, known word, and unknown word accuracies for the
test data, as well as tagging speed indications.

{\small
\begin{verbatim}
Memory Based Tagger Version 1.0
...
Done:
  4427 words processed.
  Known   words: 3635   correct from 3810 (95.4068 %)
  Unknown words: 448    correct from 617 (72.6094 %)
  Total        : 4083   correct from 4427 (92.2295 %)
  Time used: 3
  Words/sec: 1475
\end{verbatim}
}

When neither {\tt -t} nor {\tt -T} is specified, Mbt processes data from
standard input.

An additional parameter which can be used is the size of a beam
search for the most probable sequence of tags for a complete
sentence. In this case, the tagger produces a distribution of
probabilities of tags for each word (rather than just the best one),
and applies a beam search to look for the most probable sequence of
tags at a global sentence level, as opposed to a deterministic
decision for each word in the sentence independently. As a default,
beam search is off (i.e. set to beam size 1). 

\subsection{Server options}

Sometimes it may take a long time to load all the data files necessary
to start tagging, for instance when {\sc ib1} is used as the {\sc
timbl} classifier and a very large corpus was used for tagger
generation. In applications where small files of text or even
individual sentences have to be tagged at different points in time,
e.g. on demand in a web demo, it is more efficient to use Mbt as
a server. With the option {\tt -S} $<portnumber>$, a tagger server can be
set up. E.g.

{\small
\begin{verbatim}
Mbt -s eindh.data.settings -S 1999 &
\end{verbatim}
}

This sets up a tagger server on port 1999 of the local host. When
running, the server can be accessed via telnet to this port through a
special purpose client application. To prevent too many simultaneous
clients from calling the server (e.g. in a demo environment), the
option {\tt -C} $<number>$ can be used to restrict the number of
clients that can communicate with the server at the same time (default
10).


\subsection{Overwriting settings file options}

Instead of using the settings file, it is also possible to specify
each data file separately with the parameters {\tt -l}, {\tt -r}, {\tt
-k}, {\tt -u}, and {\tt -L}. These options can also be used to
override the values given in a settings file.

\section{Timbl options}

To construct a memory-based tagger and use it, TiMBL is
used. Various algorithms, parameters, and metrics are implemented in
TiMBL. By default, the memory-based learning algorithm used to
train and test a tagger is {\sc igtree} for the known words, and {\sc
ib1} with the overlap metric with gain ratio feature weighting, and k
(the number of nearest neighbors) set to 1 for the unknown words.

This is most likely not the best (or even a good) setting for the
particular corpus you want to build a tagger from.  With the {\tt -O} 
$<string>$ parameter of Mbtg, a string can be provided with
settings for the TiMBL algorithm to be used in tagger generation
and tagging. The string has the following structure:

{\small
\begin{verbatim}
-O "K: <known words options> U: <unknown words options>" 
\end{verbatim}
}

or when the options are the same for known and unknown word tagging: 

{\small
\begin{verbatim}
-O "<options>"
\end{verbatim}
}

The following is an example: 

{\small
\begin{verbatim}
-O "K: -a1 -w1 -vS U: -a0 -w1 -mM -k9 -vS" 
\end{verbatim}
}

This means that {\sc igtree} (features sorted according to information
gain, and classification through decision-tree traversal) will be used
for the known words, and {\sc ib1} with gain ratio feature weighting,
modified value difference metric, and 9 nearest neighbours, will be
used for the unknown words. The {\tt -Vs} setting mutes the output of the {\sc
timbl} classifiers. When this option is left out, the TiMBL
output is also printed to screen. Using these particular example
settings increase overall accuracy on the {\tt eindh.test} file to
92.86\%.

Please consult the reference guide provided with the TiMBL
package to experiment with other parameter settings. Note that
changing some TiMBL parameter must be accompanied by a complete
regeneration of the tagger with {\tt Mbtg}. Also, changing the
verbosity settings of TiMBL may have the effect that more output
is generated to the standard output streams than what is documented here.

\clearpage

\bibliographystyle{fullname}
\bibliography{mbtman}

\end{document}
