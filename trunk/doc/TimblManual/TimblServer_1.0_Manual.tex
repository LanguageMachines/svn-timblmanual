% TiMBLServer 1.0 manual

\documentclass{report}

\usepackage{epsf}
\usepackage{epsfig}
\usepackage{a4wide}
\usepackage{palatino}
\usepackage{fullname}
\usepackage{url}

\newcommand{\chisq}{{$ \chi^2 $}}

\author{Ko van der Sloot \\
	Induction of Linguistic Knowledge Research Group\\
	Tilburg centre for Creative Computing\\
        Tilburg University \\ \\
       URL: http://ilk.uvt.nl\thanks{This document is available from
	http://ilk.uvt.nl/downloads/pub/papers/ilk.1002.pdf. All rights reserved
	Induction of Linguistic Knowledge, Tilburg University and 
        CLiPS, University of Antwerp.}}

\title{{\huge TimblServer: Tilburg Memory-Based Learner Server} \\ \vspace*{0.5cm}
{\bf version 1.0} \\ \vspace*{0.5cm}{\huge Manual}\\
\vspace*{1cm} {\it ILK Technical Report -- ILK-10-02}}

%better paragraph indentation
\parindent 0pt
\parskip 9pt

\begin{document}

\pagenumbering{roman} 

\maketitle

\tableofcontents

\chapter*{Preface}

This reference guide describes the server interface to TiMBL, the
Tilburg Memory-Based Learner software package. To be able to use
TimblServer a working installation of TiMBL (version 6.3 or upward)
should be present on your system. For information on the installation
and use of TiMBL, see the TiMBL Reference Guide \cite{Daelemans+10}
and the TiMBL webpage\footnote{\url{http://ilk.uvt.nl/timbl}}.

This is the first release of the separate TimblServer package, which
includes some new functionality compared to TiMBL 6.2 and earlier
versions.  Although this release has been tested for some time in our
research groups, the software may still contain bugs and
inconsistencies in some places. We would appreciate it if you would
send bug reports, ideas about enhancements of the software and the
manual, and any other comments you might have, to {\tt Timbl@uvt.nl}.

This reference guide is structured as follows. In
Chapter~\ref{license} you can find the terms of the license according
to which you are allowed to use TimblServer. The subsequent chapter gives
some instructions on how to install the TimblServer package on your
computer. Chapter~\ref{changes} lists the changes that have taken
place up to the current version. Next, Chapter~\ref{tutorial} offers a
quick-start tutorial for readers who want to get to work with TimblServer
right away. The tutorial describes, step-by-step, a case study with a
sample data set (included with the TiMBL software) representing the
linguistic domain of predicting the diminutive inflection of Dutch
nouns. 

\chapter{GNU General Public License}
\label{license}
\pagenumbering{arabic} 

TimblServer is free software; you can redistribute it and/or modify it under the terms of the GNU General Public License as published by the Free Software Foundation; either version 3 of the License, or (at your option) any later version.

TimblServer is distributed in the hope that it will be useful, but WITHOUT ANY WARRANTY; without even the implied warranty of MERCHANTABILITY or FITNESS FOR A PARTICULAR PURPOSE.  See the GNU General Public License for more details.

You should have received a copy of the GNU General Public License along with TiMBL.  If not, see $<$http://www.gnu.org/licenses/$>$.

In publication of research that makes use of TimblServer 1.0, a citation should be given of: {\em ``Walter Daelemans, Jakub Zavrel, Ko van der
  Sloot, and Antal van den Bosch (2009). TiMBL: Tilburg Memory Based
  Learner, version 6.3, Reference Guide. ILK Technical Report ILK-1001
  Available from \\ {\tt
    http://ilk.uvt.nl/downloads/pub/papers/ilk1001.pdf}''}

\pagestyle{headings}

\chapter{Installation}

%TODO: Uitleg over (debian) packages?

\vspace{-1cm}
The TimblServer package can be obtained as a gzipped tar archive from:

{\tt http://ilk.uvt.nl/timbl}

Following the links from that page, you can download the file {\tt
  timblserver-1.0.tar.gz}. This file contains the complete source code
(C++) for the TimblServer program, the license, and documentation. The
installation should be relatively straightforward on most UNIX
systems.

Before installing TimblServer, please make sure that a recent Timbl
version ($\geq 6.3$) is installed on your system. (If in doubt, check
the output of the command {\tt Timbl -V}).

To install the package on your computer, unzip the downloaded file ({\tt >} is the command line prompt):

{\tt > tar xfz timblserver-1.0.tar.gz}

This will make a directory {\tt timblserver-1.0} under your current directory.

Alternatively you can do:

{\tt > gunzip timblserver-1.0.tar.gz}

and unpack the tar archive:

{\tt > tar xf timblserver-1.0.tar}

Go to the timblserver-1.0 directory, and configure the package by typing

{\tt > cd timblserver-1.0} \\
{\tt > ./configure --prefix=<location\_to\_install>}

If you do not use the {\tt --prefix} option, TimblServer will try to
install itself in the directory {\tt /usr/local/}.  If you do not have
{\tt root} access you can specify a different installation location
such as {\tt \$HOME/install}

It is not obligatory to install TimblServer, but if you plan to
install TiMBL-based extensions such as
Mbt\footnote{\url{http://ilk.uvt.nl/mbt}} or
Tadpole\footnote{\url{http://ilk.uvt.nl/tadpole}}, installing
TimblServer is recommended.

After {\tt configure} you can build TimlbServer

{\tt > make}

and (as recommended) install:

{\tt > make install }

If the process was completed successfully, you should now have
executable files named {\tt TimblServer} and {\tt TimblClient} in the
directory {\tt <location\_to\_install>/bin}, and a static library {\tt
  libTimblServer.a} in the directory {\tt
  <location\_to\_install>/lib}.

Within the {\tt <location\_to\_install>} directory a subdirectory is
also created: {\tt share/doc/timblserver} where the TimblServer 1.0
documentation can be found.

TimblServer should now be ready for use. If you want to run the
examples and demos from this manual, you should act as follows:

\begin{itemize}
\item Be sure to add {\tt <location\_to\_install>/bin} to your PATH. 
In many shells something like

 {\tt > export PATH=\$PATH:<location\_to\_install>/bin }
will do.
\item We will use the same examples as TiMBL, so copy all the files from {\tt
  <location\_to\_install>/share/doc/timbl/examples} to some working
location. (By default, TiMBL writes its results to the directory where
it finds the data.)
\item and test:

{\tt cd} to the working location, and then

{\tt > TimblServer -f dimin.train -S 7000}

This will start a TimblServer on port 7000 (if that port is occupied you might need to try another number).

The server can be used (from the same machine) over Telnet like this:

{\tt > telnet localhost 7000 }

The server responds:

{\tt Welcome to the Timbl server.}

Try to classify an instance:

{\tt c =,=,=,=,=,=,=,=,+,p,e,=,? }

The server should respond with:

{\tt CATEGORY \{T\} }

\end{itemize}

The e-mail address for problems with the installation, bug reports, comments and questions is {\tt Timbl@uvt.nl}.

\chapter{Changes}
\label{changes}

\section{version 1.0}

This is the initial release of a separate TimblServer.
Compared with the version included in Timbl (up to version 6.2) the following is changed:

\begin{itemize}

\item Some bugs are fixed ({\tt -C <n>} didn't work as expected)
\item a {\tt --config} option is added to read a configuration file. This
  file can be used to specify a whole range of separated and unrelated
  TiMBL experiments all to run on the same TCP port.

\end{itemize}


\chapter{Quick-start Tutorial}
\label{tutorial}

The use of TimblServer subsumes a familiarity with TiMBL.  The
TimblServer commandline interface is the same as that of Timbl with a
few additions.


\chapter{Software usage and options}
\label{reference}

\section{Command line options}
\label{commandline}

The user interacts with TimblServer through the use of command line
arguments.  When you have installed TimblServer successfully, and you
type {\tt TimblServer} at the command line without any further
arguments, it will print an overview of the most basic command line
options.

{\footnotesize
\begin{verbatim}
TiMBL Server 1.0.0 (c) ILK 1998 - 2010.
Tilburg Memory Based Learner
Induction of Linguistic Knowledge Research Group, Tilburg University
CLiPS Computational Linguistics Group, University of Antwerp
Tue Mar 16 14:08:22 2010

usage:  TimblServer --config=config-file
or      TimblServer -f data-file {-S socket} {-C num}
or see: TimblServer -h
        for more options
\end{verbatim}
}

\subsection{server options}

\begin{description}

\item {\tt --config <file>} : Start a server with a range of TiMBL experiments as specified in the configfile.

This is the {\bf prefered} way to use TimblServer. See Section~\ref{configfile}. for details about the configfile.

\item {\tt -S <portnumber>} : Starts a TiMBL server listening on the
  specified port number of the localhost.  This option is only
  available to be backward compatible with Timbl 6.2 and earlier
  versions; the {\tt --config} variant is preferred.
\item {\tt -C <number>} : limit the number of parallel connections to
  {\tt <number>}. The default is 25.

\end{description}

\subsection{configuration file}
\label{configfile}
The configfile contains lines of the general format {\tt
  attribute=value}.  Valid attributes are: {\tt port} to set the TCP
port and {\tt maxconn} to set the numer of parallel connections of the
server. The attribute {\tt protocol} is reserved for future usage. (We
plan to implement a HTTP interface).

Besides those attribute/value pairs, the configfile can contain any
number of lines describing TiMBL experiments to be served. The lines
look like {\tt basename=''timbl training options''}.  {\tt basename}
is a name you can choose to identfy the experiment. The {\tt basename}
is needed when you call the server to classify an instance.

{\tt ''timbl training options''} is a string of options which are used
to train Timbl. For example:

\begin{verbatim}
port=7000
maxconn=10
dimin0="-aIB1 +vdi+db+n -f dimin.train"
dimin1="-aIGTREE +vdi+db +D -f dimin.train"
dimin2="-aTRIBL +vdi+db+n -q3 -f dimin.train"
\end{verbatim}

If you start TimblServer with this configfile, it will start on port
7000, with at most 10 parallel sessions. The server will train 3
different Timbl instances, identified by the names {\tt dimin0}, {\tt
  dimin1} and {\tt dimin2}.

\section{Server interface}
\label{serverformat}

When the TimblServer is running, it is waiting for input on the
specified port number. When a client connects on this port number, the
server starts a separate thread, processing any given commands, such
as to classify a new example. A sample client program is included in
the distribution. The client must communicate with the server using
the protocol described below. After accepting the connection, the
server first sends a welcome message to the client:

{\tt Welcome to the Timbl server.}

After this, the server waits for client-side requests.  The client can
now issue five types of commands: {\tt base}, {\tt classify}, {\tt set} (options), {\tt query} (status), and {\tt exit}. The type of command is specified by the the first string of the request line, which can be abbreviated to any prefix of
the command, up to one letter (i.e.~b, c,s,q,e). The command is
followed by whitespace and the remainder of the command as described
below.

\begin{description}
\item {\tt base basename}\\
  tell the server to which base the next command(s) refer. The
  basename must be one of the basenames specified in the {\em
    configfile}.  If {\bf NO} configfile was specified (that means:
  TimblServer is started with the old-school -S option) then the
  basename is ``default''. It is not needed to specify that when
  testing. (therefore preserving backward compatability with older
  Timbl versions)

\item {\tt classify testcase}\\
  testcase is a pattern of features (must have the same number of
  features as the training set) followed by a category
  string. E.g.: {\tt small,long,1,??.}\\
  Depending on the current settings of the server, it will either
  return the answer
      \begin{verbatim}
	ERROR { explanation }
      \end{verbatim}
      if something's gone wrong, or the answer
      \begin{verbatim}
CATEGORY {category} DISTRIBUTION { category 1 } DISTANCE { 1.000000} NEIGHBORS
ENDNEIGHBORS
      \end{verbatim}
      where the presence of the {\tt DISTRIBUTION}, {\tt DISTANCE} and
      {\tt NEIGHBORS} parts depends upon the current verbosity
      setting. Note that if the last string on the answer line is {\tt
        NEIGHBORS}, the server will proceed to return lines of nearest
      neighbor information until the keyword {\tt ENDNEIGHBORS}.
\item {\tt set option}\\ where option is specified as a string of
      commandline options (described in detail in
      Chapter~\ref{commandline} below). Only the following commandline
      options are valid in this context: {\tt k, m, d, B, L, Q, w, v,
      x}. The setting of an option in this client does not affect the
      behavior of the server towards other clients. The server replies
      either with {\tt OK} or with {\tt ERROR \{explanation\}}.
    \item {\tt query}\\
      queries the server for a list of current settings. Returns a
      number of lines with status information, starting with a line
      that says {\tt STATUS}, and ending with a line that says {\tt
        ENDSTATUS}. For example:

\begin{footnotesize}
\begin{verbatim}
STATUS
FLENGTH              : 0
MAXBESTS             : 500
NULL_VALUE           : 
TREE_ORDER           : G/V
DECAY                : Z
INPUTFORMAT          : Column
SEED                 : -1
DECAYPARAM           : 1.000000
SAMPLE_WEIGHTS       : -
IGNORE_SAMPLES       : +
PROBALISTIC          : -
VERBOSITY            : F
EXACT_MATCH          : -
USE_INVERTED         : -
GLOBAL_METRIC        : Overlap
METRICS              : 
NEIGHBORS            : 1
PROGRESS             : 100000
TRIBL_OFFSET         : 0
IB2_OFFSET           : 0
WEIGHTING            : GRW
ENDSTATUS
\end{verbatim}
\end{footnotesize}

\item {\tt exit}\\
  closes the connection between this client and the server.
\end{description}

\clearpage

\bibliographystyle{fullname}
\bibliography{../../ilkbib/ilk}

\end{document}
